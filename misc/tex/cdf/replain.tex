% This file is part of the Digital Assyriologist.  Copyright
% (c) Steve Tinney, 1994, 1995.  It is distributed under the
% Gnu General Public License as specified in /da/doc/COPYING.
%
% $Id: replain.tex,v 0.4 1996/05/27 16:19:21 s Exp $

% Redo or extend some of the things that plain has done. Cleaner than
% merging it with daplain.tex

% Shifting catcodes
\chardef\lettercat=11
\chardef\othercat=12
\def\makeother#1{\catcode`#1\othercat\relax}
\def\makeletter#1{\catcode`#1\lettercat\relax}
\def\makeactive#1{\catcode`#1\active\relax}
\def\makeactiveB#1{\catcode#1\active\relax}

% We make @ signs act like letters (again) to avoid conflict 
% between user names and internal control sequences of the DA format.
\makeletter{@}

\catcode`\^^K=14 % Sprint ruler lines are like %-signs
\catcode`\^^L=\active \outer\def^^L{\par} % ascii form-feed is "\outer\par"
\catcode`\^^Z=9 % DOS EOF marker is ignored
\catcode`\^^_=10 % Sprint SoftNL is just a space
\catcode`\^^\=\active \let\^^\~ % hard space in Sprint is a non-breaking space

\def^^L{\par} % remove the outer used by Plain

% change some parameter values
\hbadness=5000 % most of the work done by this format will be drafting
\vbadness=5000
\clubpenalty=10000
\widowpenalty=10000
\displaywidowpenalty=2000
\newlinechar=`\^^J
\errorcontextlines=0
\hfuzz=2pt
\vfuzz=2pt

% A minor tweak to Plain's narrower, to make it consistent with
% everything else.
\def\narrower{\advance\leftskip\indention\advance\rightskip\indention}
\def\wider{\advance\leftskip-\indention\advance\rightskip-\indention}

% The DA manuscript format has a few important differences from the 
% macro writing format:
% The quote marks are special since they force a roman font to be used
\newif\ifinquotes 
\def\inquoteshook{}
{\makeactive{"}\gdef"{\ifinquotes\egroup\else\bgroup\rm\inquotestrue\inquoteshook\fi}}
% the underline character is used as another form of visible space
{\makeactive{_}\gdef_{~}}
\def\DAcatcodes{\makeactive{_}\makeactive{"}\catcode"40=0\relax}%\makeactive{.}
% This is the list of special DA characters
%{\DAcatcodes\gdef\DAdospecials{\do_\do.\do"}}
\def\DAdospecials{\do\_\do.\do"\do|}

% more prepackaged trace options, a la \tracingall, and a new \tracingall
% that traces paths and font loads
\def\traceoutput{\tracingoutput=1 \showboxbreadth10000 \showboxdepth10000 }
\def\tracemacros{\tracingmacros=1 }
\def\tracingalot{\tracingonline0 \tracingcommands\tw@\tracingstats0
  \tracingpages\@ne\tracingoutput\@ne\tracinglostchars\@ne
  \tracingmacros\tw@\tracingrestores\@ne\tracingparagraphs0
  \showboxbreadth\maxdimen\showboxdepth\maxdimen\batchmode}
\def\tracingall{\tracingonline\@ne\tracingcommands\tw@\tracingstats\tw@
  \tracingpages\@ne\tracingoutput\@ne\tracinglostchars\@ne
  \tracingmacros\tw@\tracingparagraphs\@ne\tracingrestores\@ne
  \showboxbreadth\maxdimen\showboxdepth\maxdimen\errorstopmode
  \tracingfonts\tw@ \tracingpaths\tw@}

% plain has \goodbreak
\def\badbreak{\par\penalty500 }

% abbreviations for some of plain's spacing commands
\let\>\thinspace \let\<\negthinspace \let\,\enskip \let\;\quad \let\:\qquad
\let\mathcomma\, \def\,{$\mathcomma$}

% write to log and screen, cf. plain's \wlog
\def\wstdout{\immediate\write16}
\endinput

\def\everymathextra{}

%%%%%%%%%%%%%%%%%%%%%%%%%%%%%%%%%%%%%%%%%%%%%%%%%%%%%%%%%

$Log: replain.tex,v $
Revision 0.4  1996/05/27 16:19:21  s
*** empty log message ***

% Revision 0.3  1996/01/18  12:09:54  s
% add \inquoteshook to be run after \rm in "..."
%
% Revision 0.2  1995/08/04  13:50:35  s
% added `|' to list of characters in \DAdospecials
%
% Revision 0.1  1994/12/12  03:56:05  s
% initial checkin
%
