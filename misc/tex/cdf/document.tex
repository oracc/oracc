% This file is part of the Digital Assyriologist.  Copyright
% (c) Steve Tinney, 1994, 1995.  It is distributed under the
% Gnu General Public License as specified in /da/doc/COPYING.
%
% $Id: document.tex,v 0.3 1996/05/27 16:18:34 s Exp sjt $

% Macros to support document definition.
%
% This file defines the macros which give documents structure.
% Titles, chapter and heading divisions, indexing and cross-referencing,
% table-of-contents production and so on. Organisation is pretty much
% front to back, simultaneously defining features useful for editions,
% papers and books.

% 1) Font switching at the document level
% =======================================
%
% Three main user commands are defined:
%    \documentsize
%    \localdocsize
%    \definesize
% The first may be used only once, before any typesetting has taken
% place, since it uses magnification to achieve its effect. The 
% second is a variant of the first which can be used to provide local
% differences in the fonts used in a document (e.g. an appendix set in 
% smaller type). The third is used to redefine a given major type size,
% either title, text or footnote. Besides these three, a full range of
% xxxpoint commands are defined which simply expand to \definesize macros
% in which the first argument is the current type size. Finally the
% means to set the current environment are provided with the macros
%    \titlesize
%    \textsize
%    \fnotesize
% which are defined automatically when a \definesize command is given.
%
% The macros are presented here in top-down sequence.

% The highest level command is that which sets the document size.
% This involves setting a titlesize, textsize and footnotesize, and
% establishing the value of leading parameters. It doesn't affect the
% size of the typefield, header location etc.
%
% We understand 10, 11 and 12 point for document sizes, and since
% 10 point is preloaded we can simply set a magnification factor
% if the user can do dynamic magnification. Otherwise we need to
% load the relevant fonts and set the environment.
\def\documentsize#1{}%\magnifydocument{#1}}
%  \ifnomag\def\next{\d@csize{#1}}\else\def\next{\magnifydocument{#1}}\fi
%  \next}

% Entry point for local resetting of type sizes without using magnification
\def\localdocsize#1{\d@csize{#1}}

% This macro always sets the document in 10pt and relies on .dvi 
% magnification to handle size shifts
\def\magnifydocument#1{\count255=#1 \advance\count255-10
  \mag
  \ifcase\count255 1000\or 1095\or 1200\or 1440\or 1728\or 2074\or 2488\fi
  \relax\d@csize{10}}

% one nicety in this macro is that by defining \textsize last we avoid the
% need to reset the size explicitly.
\def\d@csize#1{%
  \ifnum#1=12 \def\next{%
      \definesize{\titlesize}{17}%
      \definesize{\fnotesize}{10}%
      \definesize{\textsize}{12}}%
  \else
    \ifnum#1=11 \def\next{%
      \definesize{\titlesize}{14}%
      \definesize{\fnotesize}{9}%
      \definesize{\textsize}{11}}%
    \else
      \ifnum#1=10 \def\next{%
        \definesize{\titlesize}{12}%
	\definesize{\fnotesize}{8}%
        \definesize{\textsize}{10}}%
      \else
        \errhelp=
	{The permitted sizes for \string\documentsize\space and \string\localdocsize^^J%
	 are 10, 11 or 12. I'm going to use 10 now, just to get us through this run.}%
        \errmessage{Illegal size for document size (#1)---using 10 instead}%
        \def\next{%
          \definesize{\titlesize}{12}%
  	  \definesize{\fnotesize}{8}%
          \definesize{\textsize}{10}}%
      \fi
    \fi
  \fi
  \next
  \textsize\unem} % initialize to normal size and set up \sum and \akk

% This is the primary entry point for resetting the font environment
% locally or globally. One can say, e.g., 
%   \definesize\titlesize{12}
% to set the text size for the given environment size. The command
% sequence #1 is defined to invoke, say, \@12, which will set up 12
% point text size with whatever was associated with that in the 
% current \FLODsetsizes command.
\def\definesize#1#2{\edef\next{\csname FLOD#2\endcsname}%
  \edef#1{\expandafter\noexpand\csname\next\endcsname\noexpand\rm\noexpand\setleading}}

% 2) Document Preamble
% ====================

% The principle for preamble is that a document typically has a 
% collection of variable attributes which are set before 
% processing begins. The variables currently available are:
%	\title
%	\author
%	\affiliation
%	\paperinfo
%	\editor
%	\publisher
%	\placeofpublication
%	\year
%	\draftnumber
%	\date
%	\dedication
% These are all defined as \toks registers, so they can be set by the
% user just by saying, e.g., \title{The Title}
\newtoks\title
\newtoks\subtitle
\newtoks\author
\newtoks\affiliation
\newtoks\paperinfo
\newtoks\editor
\newtoks\publisher
\newtoks\placeofpublication
\newtoks\yearofpublication
\newtoks\drafttitle
\newtoks\draftnumber
\newtoks\date
\newtoks\dedication
\newtoks\epigraph

\def\draftheading{\lefthdrinfo{\the\drafttitle\ \the\draftnumber}\centerhdrinfo{\datestamp}}

% After the user has set up the variables there are several predefined 
% formats for utilising their contents.
% 
% Simplest is the paper which has a title at its head, then gives the
% author, affiliation and any miscellaneous information such as the
% reason the paper is being written, where it will be published, etc.
%
\def\titleheader{\titleheadersizes
  \vglue\abovetitleskip
  {\titleformat\the\title\par}%
  \testtoks\subtitle
  \ifemptytoks\else\vglue\intertitleskip\fi
  {\subtitleformat\the\subtitle\par}%
  \testtoks\title
  \ifemptytoks\else\vglue\belowtitleskip\fi
  {\authorformat\the\author\par}%
  \testtoks\author
  \ifemptytoks\else\vglue\belowauthorskip\fi
  {\affiliationformat\the\affiliation\par}%
  \testtoks\affiliation
  \ifemptytoks\else\vglue\belowaffiliationskip\fi
  {\paperinfoformat\the\paperinfo\break\the\date\par}%
  \vglue\belowpaperinfoskip
  \testtoks\epigraph
  \ifemptytoks\else
    {\epigraphformat\the\epigraph\par}\vglue\belowepigraphskip\fi
}
\newif\ifemptytoks
\def\testtoks#1{%
  \def\next{\the#1}\ifx\next\empty\emptytokstrue\else\emptytoksfalse\fi}
\newskip\abovetitleskip
\newskip\intertitleskip
\newskip\belowtitleskip
\newskip\belowauthorskip
\newskip\belowaffiliationskip
\newskip\belowpaperinfoskip
\newskip\belowepigraphskip
\def\titleformat{\centerpar\titlesize\bem}
\def\subtitleformat{\centerpar\textsize\bem}
\def\authorformat{\centerpar}
\def\epigraphformat{\rightpar}
\def\affiliationformat{\centerpar\em}
\def\paperinfoformat{\leftskip.5\hsize\rightpar}
\def\titleheadersizes{
  \abovetitleskip4pc
  \intertitleskip1pc
  \belowtitleskip2pc	
  \belowauthorskip1pc
  \belowaffiliationskip2pc
  \belowepigraphskip2pc
  \belowpaperinfoskip4pc\relax}

% next simplest is the paper with a title page
\def\titlepage{\titlepagesizes
  \nonumberthispage
  \vskip\abovetitleskip
  {\titleformat\the\title\par}%
  \vskip\belowtitleskip
  {\authorformat\the\author\par}%
  \vskip\belowauthorskip
  {\affiliationformat\the\affiliation\par}%
  \vskip\belowaffiliationskip
  {\paperinfoformat\the\paperinfo\par}%
  \vskip\belowpaperinfoskip
  \newpage
}
\def\titlepagesizes{
  \abovetitleskip2in plus1fill
  \belowtitleskip1in	
  \belowauthorskip.5in
  \belowaffiliationskip1in
  \belowpaperinfoskip0pt plus1fill\relax}

%% more complicated is the use of several pages of frontmatter
%\begin{frontmatter}
%\titlepage
%\halftitlepage
%\copyrightpage
%\preface
%\listofabbreviations
%\tableofcontents
%\listofplates
%\listoftables
%\listoffigures
%\end{frontmatter}
%
%\epigraph ...		% multiple epigraphs should be OK
%\attribution... \par	% the attribution should be optional
%

\def\endfrontmatter{\newpage
  % ensure front matter has even number of pages
  \ifodd\pageno\else\blankpage\fi 
  \pageno1\relax}
%  \rightpagechapterstrue

\def\frontnote#1{\vbox to.9\vsize{\vfill
  \centerline{[#1 = page \romannumeral-\pageno]}\vfill}}

\input div
\input toc
\input xref

\endinput

%%%%%%%%%%%%%%%%%%%%%%%%%%%%%%%%%%%%%%%%%%%%%%%%%%%%%%%%%

$Log: document.tex,v $
Revision 0.3  1996/05/27 16:18:34  s
*** empty log message ***

% Revision 0.2  1996/02/20  01:27:24  s
% futz with \draftheading and \paperinfoformat
%
% Revision 0.1  1994/12/12  03:56:05  s
% initial checkin
%
