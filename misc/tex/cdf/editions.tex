% This file is part of the Digital Assyriologist.  Copyright
% (c) Steve Tinney, 1994, 1995.  It is distributed under the
% Gnu General Public License as specified in /da/doc/COPYING.
%
% $Id: editions.tex,v 0.21 1996/09/20 01:57:09 s Exp sjt $

% This file defines all the macros directly connected with producing 
% editions of texts. It is arranged in a more or less top-down fashion,
% much as one would encounter the various parts of a model edition.

\newif\if@composition	\newtoks\composition@toks
\newif\if@longname	\newtoks\longname@toks
\newif\if@citeas	\newtoks\citeas@toks
\newif\if@PSDname	\newtoks\PSDname@toks
\newif\if@text		\newtoks\text@toks

% At the very top level are three macros which are not currently used
% by DA-TeX (though they might be in the future).
\def\literary{}
\def\lexical{}
\def\economic{}

% Every edition begins with an @composition{} or an @text{}.
\def\composition#1{\def\next{#1}%
  \ifx\empty\next\@compositionfalse
  \else\@compositiontrue\composition@toks{#1}\fi
  \@textfalse\comp@startedfalse
  \compositionhdr}
\let\edition\composition % backward compatibility
\def\text#1{\def\next{#1}%
  \ifx\empty\next\@textfalse
  \else\@texttrue\text@toks{#1}\fi
  \comp@startedfalse}
\def\longname#1{\def\next{#1}%
  \ifx\empty\next\@longnamefalse
  \else\@longnametrue\longname@toks{#1}\fi}
\def\citeas#1{\def\next{#1}%
  \ifx\empty\next\@citeasfalse
  \else\@citeastrue\citeas@toks{#1}\fi}
\def\PSDname#1{\def\next{#1}%
  \ifx\empty\next\@PSDnamefalse
  \else\@PSDnametrue\PSDname@toks{#1}\fi}
\def\start@composition{\def\next{\st@rt@composition}%
  \if@composition
    \else\if@text
        \else\def\next{\relax}\fi\fi
  \ifcomp@started\def\next{\relax}\fi
  \next}
\newif\ifcomp@started
\def\st@rt@composition{\comp@startedtrue
  \ifnum\pageno>1\newpage\fi
  \if@composition
    \if@longname\def\next{\heading{\the\composition@toks\ (\the\longname@toks)}}%
    \else\def\next{\heading{\the\composition@toks}}\fi
  \else\def\next{\heading{\the\text@toks}}\fi
  \next}

% After the composition is begun several optional sections of text are
% supported. The first of these is the textnotes environment, which is
% intended to provide a home for miscellaneous preamble concerning the
% edition.
\def\textnotes{\textnotehdr\parindent\indention \parskip2pt plus1pt}
\def\endtextnotes{}

% A home for crediting others (and oneself!)
\def\credits{\creditshdr\parindent\indention \parskip2pt plus1pt}

% The next is a place to list previous bibliography in an informal and
% possibly annotated manner
\def\prevlit{\prevlithdr \nextpar{\hangindent\indention} 
  \parindent0pt \parskip2pt plus1pt \relax}
\def\endprevlit{}

% The heading for each section can be redefined easily
\def\compositionhdr{\heading\the\composition@toks\par}
\def\textnotehdr{\heading\introductionstr\par}
\def\creditshdr{\heading\creditsstr\par}
\def\prevlithdr{\heading\previousliteraturestr\par}

% for backward compatibility we define a control sequence that
% places its argument in roman 
\def\obs#1{{\rm#1}}

% When more than one text is being used the @text command is not
% appropriate. Instead we define a way of presenting the sources
% in a descriptive environment in which each element is accorded its
% own label. This is divided into two parts, the first being the 
% sources environment, the second the macros used for producing bar
% diagrams of the lines of the composition preserved by each source.
%
%%%%%%%%%%%%%%%%%%%%%%%%%%%%%%%%%%%%%%%%%%%%%%%%%%%%%%%%%%%%%%%%%%%
%%		PART ONE: SOURCE LISTS
%%%%%%%%%%%%%%%%%%%%%%%%%%%%%%%%%%%%%%%%%%%%%%%%%%%%%%%%%%%%%%%%%%%
%%
%% The siglum macro is the key to the regular sources table, and
%% can be redefined to change the organisation there.
\newbox\siglumb@x \newbox\musnob@x \newbox\linesb@x \newbox\descb@x
\newbox\bibb@x    \newbox\provb@x  \newbox\typeb@x  \newbox\ltypeb@x
\newbox\withb@x
\newdimen\Sousiglumwd	\newdimen\Soumusnowd	\newdimen\Soubibwd
\newdimen\Souprovwd	\newdimen\Soulineswd	\newdimen\Soudescwd
\newdimen\Souhangindent \Souhangindent0pt
\newdimen\Souparindent  \Souparindent0pt 
\newskip\Sougutter	
\newskip\Souintersourceskip  \Souintersourceskip2pt plus3pt
\newif\iffirstsource
\newif\ifgivingprov	\givingprovfalse

\newif\ifpara@format
\def\paraformat{\para@formattrue}
\def\tableformat{\para@formatfalse}

\newif\ifsources
\newif\ifmoresources
\def\sources{\tableformat \disabledots
  \callsetsouwds \parindent\Souparindent \global\firstsourcetrue
  \sourcestrue}
\def\endsources{}
\def\defaultsouwds{\setsouwds{.05}{.275}{0}{.05}{.3}{.225}{1pt plus1fil}}
\def\callsetsouwds{\defaultsouwds}

\def\giveprovenience{\givingprovtrue}
\def\omitprovenience{\givingprovfalse}

\def\setsouwds#1#2#3#4#5#6#7{%
  \Sousiglumwd#1\hsize
  \Soumusnowd#2\hsize
  \Soubibwd#3\hsize
  \Souprovwd#4\hsize
  \Soudescwd#5\hsize
  \Soulineswd#6\hsize
  \Sougutter#7
  \ifgivingprov\else
    \ifdim\Souprovwd>0pt 
      \dimen0=\Souprovwd \divide\dimen0by3
      \Souprovwd0pt
      \advance\Soumusnowd\dimen0
      \advance\Soudescwd\dimen0
      \advance\Soulineswd\dimen0
    \fi
  \fi
}

\def\dob@xes{\do{siglum}\do{musno}\do{bib}\do{prov}\do{lines}%
             \do{type}\do{ltype}\do{with}\do{desc}}
\def\emptyb@xes{\def\do##1{\setbox\csname##1b@x\endcsname\hbox{}}\dob@xes}
\def\siglum{\emptyb@xes\setbox\siglumb@x\hbox\bgroup}
\def\musno{\egroup\setbox\musnob@x\hbox\bgroup}
\def\bib{\egroup\setbox\bibb@x\hbox\bgroup}
\def\prov{\egroup\setbox\provb@x\hbox\bgroup}
\def\lines{\egroup\setbox\linesb@x\hbox\bgroup}
\def\type{\egroup\setbox\typeb@x\hbox\bgroup}
\def\ltype{\egroup\setbox\ltypeb@x\hbox\bgroup}
\def\with{\egroup\setbox\withb@x\hbox\bgroup}
\def\desc{\egroup\setbox\descb@x\hbox\bgroup}
\def\done{\egroup\formatsiglum}
\let\exemplar\siglum
\let\from\prov
\let\pub\bib

\def\soubox#1{\setbox0\hbox{\ignorespaces\unhbox\csname#1b@x\endcsname\unskip}}
\def\putclean#1{\ignorespaces#1\unskip}
\def\Soupar{\leavevmode \indent \hangindent\Souhangindent \rightskip.5em plus1fil \nohyphenation\relax}

\def\basicsourceformat{\line{\hskip\Sougutter%
  \hbox to\Sousiglumwd{\hss\putclean{\strut\unhbox\siglumb@x}\hss}%
  \hskip\Sougutter
  \vtop{\hsize\Soumusnowd \Soupar \strut \ignorespaces
    \soubox{musno}\def\next{}%
    \ifdim\wd0>0pt \unhbox0\ %
      \soubox{bib}\ifdim\wd0>0pt \def\next{)}(\fi
    \fi
    \putclean{\unhbox0}%
    \next
    \unskip\strut}%
  \hskip\Sougutter
  \ifgivingprov
    \vtop{\hsize\Souprovwd \Soupar \putclean{\unhbox\provb@x}\strut}\hskip\Sougutter
  \fi  
  \vtop{\hsize\Soudescwd \Soupar \putclean{\unhbox\descb@x}\strut}%
  \hskip\Sougutter
  \vtop{\hsize\Soulineswd \Soupar \putclean{\unhbox\linesb@x}\strut}%
  }}

\def\basicsourceheader{\vbox{\hrule\kern3pt\line{\hskip\Sougutter%
  \hbox to\Sousiglumwd{\hss\it\sourcestr\hss}%
  \hskip\Sougutter
  \vbox{\hbox to\Soumusnowd{\hss\it\museumnumbersstr\hss}%
        \hbox to\Soumusnowd{\hss\it\andpublicationstr\hss}}%
  \hskip\Sougutter
  \ifgivingprov \hbox to\Souprovwd{\hss\it\provstr\hss}\hskip\Sougutter \fi
  \hbox to\Soudescwd{\hss\it\descriptionstr\hss}%
  \hskip\Sougutter
  \hbox to\Soulineswd{\hss\it\linesstr\hss}}\kern3pt\hrule}}

\def\basicparasourceformat{\par\noindent\hangindent\Sousiglumwd
  \hbox to\Sousiglumwd{\putclean{\strut\unhbox\siglumb@x}\hfil}%
  \soubox{musno}\def\next{}%
  \ifdim\wd0>0pt \unhbox0
    \soubox{bib}\ifdim\wd0>0pt \def\next{). }\ (\else\hbox{. }\fi
  \fi
  \putclean{\unhbox0}\next\hbox{}%
  \ifgivingprov\putclean{\unhbox\provb@x. }\fi
  \putclean{\unhbox\descb@x}\space\hbox{}%
  \putclean{\unhbox\linesb@x}}

\newif\ifmoresources
\def\formatsiglum{%
  \iffirstsource\global\firstsourcefalse
    \ifmoresources\global\moresourcesfalse\fi
    \ifpara@format\let\next\relax
    \else\def\next{\MIDinsert
    	           \basicsourceheader
		   \vskip5pt
		   \endinsert}%
    \fi
  \else
    \ifmoresources\global\moresourcesfalse
         \ifpara@format\let\next\relax
         \else\def\next{\topinsert
			\basicsourceheader
			\vskip5pt
			\endinsert
			\vskip\Souintersourceskip}%
	 \fi
    \else\def\next{\vskip\Souintersourceskip}%
    \fi
  \fi
  \next
  \ifpara@format\def\next{\basicparasourceformat}%
  \else\def\next{\basicsourceformat}\fi
  \next}

%%%%%%%%%%%%%%%%%%%%%%%%%%%%%%%%%%%%%%%%%%%%%%%%%%%%%%%%%%%%%%%%%%%
%%		PART TWO: SOURCE DIAGRAMS
%%%%%%%%%%%%%%%%%%%%%%%%%%%%%%%%%%%%%%%%%%%%%%%%%%%%%%%%%%%%%%%%%%%
%%
%% these macros facilitate the drawing of bar-graph like 
%% representations of which lines are preserved in different 
%% exemplars of a text.
%%

\newdimen\exxrulewd
\newdimen\exxunit
\newdimen\exxtablesize
\newdimen\vlistheight
\newcount\currtop
\newcount\currbot
\newcount\lastbot
\newcount\startline
\newcount\endline
\newcount\countx
\newbox\vlistbox
\newbox\leftscalebox
\newbox\rightscalebox
\newif\ifneedsiglum
\newtoks\currexemplar
\newif\ifExxfontsloaded	\Exxfontsloadedfalse
\def\Exxloadfonts{%
  \global\Exxfontsloadedtrue
  \font\Exxsc@lefont=cmrr4 at7pt 
  \font\Exxsigl@mfont=cmrr4 at7pt }
\def\Exxscalefont{\Exxsc@lefont}
\def\Exxsiglumfont{\Exxsigl@mfont}

% This is merely sugar in the source file
\def\sourcediagram#1{}

\def\addsiglum#1{%
  \global\setbox\vlistbox=
    \vbox{\unvbox\vlistbox%
	\vbox to0pt{\vss%
	\setbox1=\hbox spread2pt{\hfil\kern\exxrulewd\Exxsiglumfont#1\hfil}%
	\moveleft.5\wd1\box1
	\vskip1pt}}}
\def\addrule#1{%
  \global\setbox\vlistbox=
    \vbox{\unvbox\vlistbox\tophtick\exxrule{#1}\bothtick}}
\def\addspace#1{%
  \global\setbox\vlistbox=
    \vbox{\unvbox\vlistbox\vskip#1\exxunit}}
\def\exxrule#1{% this makes the graph lines
  \hrule height#1\exxunit depth0pt width\exxrulewd}
\def\tick{\hrule height .5pt depth0pt width3pt \kern-.5pt} % for scale
\def\htick{% for top and bottom of graph lines
  \hbox to0pt{\hss\hskip\exxrulewd\vrule height.25pt depth0pt width2pt\hss}}
\def\tophtick{\htick\kern-.25pt}
\def\bothtick{\kern-.25pt\htick}

%#1=heading, #2=beginning line, #3=end line
\def\beginexemplartable#1#2#3{\endgraf % we get into vmode, and stay there
  \ifExxfontsloaded\else\Exxloadfonts\fi
  \ifdim\exxtablesize=0pt \exxtablesize\hsize \fi
  \ifdim\exxrulewd=0pt \exxrulewd.75truept \fi
  \ifdim\exxunit=0pt \exxunit1.8truept \fi% size of 1 text line
  \begingroup
    \hbox to\exxtablesize{\hss#1\hss}%
	\vskip\baselineskip % because \addsiglum uses 0 height for exemplar letter
	\bigskip
	\offinterlineskip \parskip0pt \parindent0pt
	\global\setbox1=\hbox{}%
	\global\startline#2
	\global\endline#3
	\count255=\endline \advance\count255-\startline
	\vlistheight=\count255\exxunit
	\makescale
	\global\setbox1=\hbox{\box\leftscalebox\hfil}}
\def\endexemplartable{%
  \endgroup
  \global\setbox1=\hbox to\exxtablesize{\unhbox1\box\rightscalebox}%
  \dimen0\pagegoal \advance\dimen0\pagetotal
  \dimen2\ht1 \advance\dimen2\dp1 %
  \ifdim\dimen0 > \dimen2 \else\newpage\fi
  \box1}

%% the scale is built in 3 phases, then assembled twice at the end.
%% phase 1 is the numbers, phase 2 the tickmarks, phase 3 the rule
\def\makescale{%
  \global\setbox\leftscalebox=\vbox{}%
  \global\setbox\rightscalebox=\vbox{}%
  \count255=5
  \global\setbox1=
    \vbox{\hbox{$\smash{\hbox{\Exxscalefont\number\startline}}$}\vskip4\exxunit}%
  \loop
    \global\setbox1=
      \vbox{\unvbox1\hbox{$\smash{\hbox{\Exxscalefont\number\count255}}$}\vskip5\exxunit}%
      \advance\count255by5
  \ifnum\count255<\endline
  \repeat
  \global\setbox3=\vbox{}%
  \count255=5
  \global\setbox3=\vbox{\tick\vskip4\exxunit}%
  \loop
	\global\setbox3=
	  \vbox{\unvbox3\tick\vskip5\exxunit}%
	\advance\count255by5
  \ifnum\count255<\endline
  \repeat
  \advance\count255by-5 % knock off excess
  \countx=\endline \advance\countx-\count255 % calculate remaining lines
  \global\setbox3=\vbox{\unvbox3\unskip\vskip\countx\exxunit\tick}%
  \global\setbox1=
	\vbox{\unvbox1\unskip\vskip\countx\exxunit%
	  \vbox to0pt{\hbox{\Exxscalefont\number\endline}\vss}%
	}%
  \global\setbox\vlistbox=\vbox{%
	\hrule height\vlistheight depth0pt width.5\exxrulewd}%
  \global\setbox\leftscalebox=\hbox{\copy1\copy3\copy\vlistbox}%
  \global\setbox\rightscalebox=\hbox{\box\vlistbox\box3\box1}}

%% to process \beginexemplar A 5-20,23-34,\endexemplar
\def\beginexemplar#1{%
  \currexemplar={#1}%
  \lastbot=0
  \needsiglumtrue
  \futurelet\testee\testendexemplar}
\def\testendexemplar{%
  \ifx\testee\endexemplar \let\next\finishexemplar 
  \else \let\next\buildexemplar \fi
  \next}

\def\buildexemplar#1,{%
  \parsenumber#1-\sentinel
  \ifneedsiglum
    \count255=\currtop
    \advance\count255-\startline
    \ifnum\count255 > 0 \addspace{\count255}\fi
    \addsiglum{\the\currexemplar}%
    \needsiglumfalse
  \else
    \count255=\currtop \advance\count255-\lastbot
    \ifnum\count255 > 0 \addspace{\count255}\fi
  \fi
    \count255=\currbot \advance\count255-\currtop
    \ifnum\count255=0 
    \addrule{.75}% if only 1 line we make a little blob, not nothing
    \addspace{-.75}% and must compensate for the addition
    \else
      \addrule{\count255}%
    \fi
  \global\lastbot=\currbot
  \futurelet\testee\testendexemplar}
\def\parsenumber#1-#2\sentinel{\def\next{#2}%
  \currtop=#1 
  \ifx\next\empty\currbot=\currtop\else\@setcurrbot#2\sentinel\fi
  \ifnum\currbot<0 \multiply\currbot-1 \fi}
\def\@setcurrbot#1-\sentinel{\currbot=#1 }
\def\endexemplar{\relax}

% pad the vlist with whitespace and add it to box0
\def\finishexemplar{%
  \global\setbox\vlistbox=\vbox to\vlistheight{\unvbox\vlistbox\vfill}%
  \global\setbox1=\hbox{\unhbox1\box\vlistbox\hfil}}

%%%%%%%%%%%%%%%%%%%%%%%%%%%%%%%%%%%%%%%%%%%%%%%%%%%%%%%%%%%%%%%%%%%%%%%%%
%% END OF SOURCES MACROS
%%%%%%%%%%%%%%%%%%%%%%%%%%%%%%%%%%%%%%%%%%%%%%%%%%%%%%%%%%%%%%%%%%%%%%%%%

%%%%%%%%%%%%%%%%%%%%%%%%%%%%%%%%%%%%%%%%%%%%%%%%%%%%%%%%%%%%%%%%%%%%%%%%%
%% Composite Macros: Introduction
%%%%%%%%%%%%%%%%%%%%%%%%%%%%%%%%%%%%%%%%%%%%%%%%%%%%%%%%%%%%%%%%%%%%%%%%%
% New composite text macros for DA user interface. In this version 
% the tab no longer separates columns: in future it will be possible
% to achieve the same effect by using {facingpage} inside an {Ncolumns}
% environment.
%
% Strategy: each line is scanned as an argument by the top-level macro.
%	It is then parsed and the sections saved in three token registers.
%	The syntax of the line is: 
%		<text>.<SPACE|TAB>	taken as line number
%		<text>:<SPACE|TAB>	taken as siglum
%		<text>			remainder is text of line
%	Blank lines are ignored. Lines beginning with a hyphen (-) 
%	are replace by a single page-wide ruling.
%	Lines beginning with at least an equals signs (=)
%	are replaced by a page-wide double ruling. 
%	Both of these effects can be suppressed by putting @relax before
%	any '-' or '=' which starts the line and should be treated 
%	literally.
%	
%	Formatting considerations are kept separate so that other 
%	interfaces can also use the parsed parts of the line.

% see local.tex for now.

%%%%%%%%%%%%%%%%%%%%%%%%%%%%%%%%%%%%%%%%%%%%%%%%%%%%%%%%%%%%%%%%%%%%%%%%%
%% SCORE ENVIRONMENT
%%%%%%%%%%%%%%%%%%%%%%%%%%%%%%%%%%%%%%%%%%%%%%%%%%%%%%%%%%%%%%%%%%%%%%%%%
% Macros for typesetting scores; code using these macros is generated with
% the preprocessor sco.
%
\newif\ifmesslines      \messlinesfalse
\newif\ifdiagnos        \diagnosfalse
\newif\ifinscore        \inscorefalse
\newif\ifchop		\chopfalse
 
\newcount\toobigct      \toobigct = 0
\newcount\donebits      \donebits = 0
\newcount\incontinuation \incontinuation = 0
\newcount\notect	\notect = 0
 
\newtoks\txttoks        % txttoks always holds the current txtlinenumber
\newtoks\sigtoks	% sigtoks always holds the current siglum
\newtoks\everyscore	% inserted before each score begins---usually once
\newtoks\conttoks	% inserted after a continuation line number

%%%%%%%%%%%%%%%%%%%%%%%%%%%%%%%%%%%%%%%%%%%%%%%%%%%%%%%%%%%%%%%%
%       'pre-output' option :  \diagnostics
%%%%%%%%%%%%%%%%%%%%%%%%%%%%%%%%%%%%%%%%%%%%%%%%%%%%%%%%%%%%%%%%
 
\def\diagnostics{\diagnostrue\messlinestrue\scrollmode
  \output={\global\setbox255\vbox{}\shipout\box255\global\advance\count0 by1}}
 
\def\nodiagnostics{\diagnosfalse\messlinesfalse\errorstopmode
  \global\output={\plainoutput}}

\newtoks\txtfonts@ze  \newtoks\numfonts@ze  \newtoks\scofonts@ze 
\newtoks\txtpoints@ze \newtoks\sigpoints@ze

\def\settxtfnt{\globaldefs1 \csname\txtfnt@ point\endcsname \globaldefs0 \sum }
\def\setscofnt{\globaldefs1 \csname\scofnt@ point\endcsname \globaldefs0 \sum }
\def\txtfnt{\csname\the\txtfonts@ze\endcsname\sum}
\def\numfnt{\csname\the\numfonts@ze\endcsname\bf}
\def\siglumfnt{\csname\the\txtfonts@ze\endcsname\sum}
\def\scofnt{\csname\the\scofonts@ze\endcsname\sum}

\def\setupfonts{%
	\savecurrsize
	\numfonts@ze{ten}%
	\txtfonts@ze{ten}%
	\scofonts@ze{eight}%
	\edef\txtfnt@{ten}%
	\edef\scofnt@{eight}}
\def\savecurrsize{\edef\svdfnt@{\the\FLODintsize}}
\def\restoresize{\globaldefs1 \csname\svdfnt@ point\endcsname\rm\globaldefs0 }

\conttoks={{\it(cont.)}\ }

\let\Hw\hidewidth
\let\Ms\multispan
\let\Om\omit
\let\@\span % was \! but this conflicts with ^!

\def\BeginBlock{%
	\global\advance\donebits by 1
        \ifnum\donebits = 1 % no extra space before the first one
	  \else\vskip\interscoreskip
	\fi
        \global\setbox0=\vbox\bgroup}

%% by allowing the \halign to be natural size we can meaningfully test
%% \wd0. If we use \halign to\hsize \wd0 always == \hsize, even when
%% the alignment is too big.
\def\Template{\settxtfnt \halign\bgroup\tabskip=\numgutter \scorestrut}

\def\scorestrut{\vrule height15pt depth5pt width0pt}

%\showboxbreadth10000 \showboxdepth10000

\def\EndBlock{\egroup\egroup
  \ifdim \wd0 > \hsize
     \toobig = \wd0 \advance \toobig by -\hsize
     \ifdim \toobig > \toobigfuzz
        \immediate\write15{Score block \the\txttoks\space is \the\toobig \space too wide [page \the\count0]}
        \global\advance\toobigct by 1
     \fi
  \fi
  \box0
  \filbreak
  \global\notect=0 \global\inscorefalse
  \ifmesslines \message{\the\txttoks} \fi
}

\let\BeginChopBlock\BeginBlock %???

\def\SuspendChopBlock{\egroup\egroup\box0}

\def\InterChopBlock{\nobreak\vskip\intercontskip}

\def\RebeginChopBlock{\choptrue\global\setbox0=\vbox\bgroup}

\def\ChopTemplate{\settxtfnt \halign\bgroup \tabskip=\numgutter \scorestrut}

\def\EndChopBlock{\tabskip=0pt plus1fil\cr\egroup\egroup
    \global\incontinuation=0
    \ifdim \wd0 > \hsize
       \toobig = \wd0 \advance \toobig by -\hsize
       \ifdim \toobig > \toobigfuzz
          \immediate\write15{alignment \the\txttoks\space is \the\toobig \space too wide [page \the\count0]}
          \global\advance\toobigct by 1
       \fi
    \fi
  \box0
  \filbreak
  \global\notect=0 \global\inscorefalse
  \ifmesslines \message{\the\txttoks} \fi
  \chopfalse
}

\def\scorenote{\noalign\bgroup\global\advance\notect by1
  \ifnum\notect > 1 \kern\internoteamt \else\kern\scoretonoteamt\fi
  \vbox\bgroup\hsize=\notewidth \leftskip=\noteleftmargin \rightskip2em plus2em
  \baselineskip\notebaselineskip \parindent2em \rm
  \aftergroup\nointerlineskip\aftergroup\par\aftergroup\egroup
  \let\next }

\def\Num#1{%
  \global\txttoks={#1}%
  \ifchop
    \hbox to\contfirstcolwd{%
      \kern\contblockleft\kern\numgutter{\numfnt\the\txttoks.\ \the\conttoks}\hfil}%
  \else
    \hbox to\contblockleft{\numfnt\the\txttoks.\ \hfil}%
  \fi
  \settxtfnt}

\def\Sig#1{%
  \global\sigtoks={#1}%
  \ifchop
    \hbox to\contfirstcolwd{\kern\contblockleft\kern\numgutter\siglumfnt\the\sigtoks\hfil}%
  \else
    \hbox to\contblockleft{\siglumfnt\the\sigtoks\ \hfil}%
  \fi
  \setscofnt}

\newbox\hyphbox		\newdimen\hyphwidth
\newbox\dotbox		\newdimen\dotwidth
\newbox\colonbox	\newdimen\colonwidth
\newdimen\spacewidth
\newbox\hybox
\newbox\astbox
\def\setupboxes{%
  \setbox\hyphbox\hbox{\txtfnt-}\hyphwidth\wd\hyphbox
  \setbox\dotbox\hbox{\txtfnt.}\dotwidth\wd\dotbox
  \setbox\colonbox\hbox{\txtfnt:}\colonwidth\wd\colonbox
  \ifnewfonts
    \setbox\hybox\hbox{{\scofnt\char254}}%
    \setbox\astbox\hbox{{\DACMR\eight\rm\char253}}%
  \else
    \setbox\hybox\hbox{\vrule height3.5pt depth-2.8pt width4pt}%
    \setbox\astbox\hbox{\lower4pt\hbox{\txtfnt*}}%
  \fi }

\def\Th{\tabskip\hyphwidth}
\def\Td{\tabskip\dotwidth}
\def\Tc{\tabskip\colonwidth}
\def\Ts{\tabskip\spacewidth}
\def\Te{\tabskip0pt}
\def\Ta{\tabskip0pt}
\def\Rh{\rlap{\copy\hyphbox}}
\def\Rd{\rlap{\copy\dotbox}}
\def\Rc{\rlap{\copy\colonbox}}
\let\Ra\relax
\let\Re\relax
\let\Rs\relax
\def\Lic#1#2{\rlap{\hbox to#1{{#2}\hfil}}}
\def\Cic#1#2{\rlap{\hbox to#1{\hfil{#2}\hfil}}}
\def\Ric#1#2{\rlap{\hbox to#1{\hfil{#2}}}}

%% versions of hyphen and asterisk for use alone in their own columns
\def\Hy{\Hw\unhcopy\hybox\Hw}
\def\Ast{\Hw\unhcopy\astbox\Hw}
%% versions of hyphen and asterisk for use when they have an lpend or rpend
%% with them
\def\HY{\unhcopy\hybox}
\def\AST{\unhcopy\astbox}

\def\Thin{}
\def\Space#1{\hbox to#1{\hfil}}
\def\DefSpace{10}

%%%%%%%%%%%%%%%%%%%%%%%%%%%%%%%%%%%%%%%%%%%%%%%%%%%%%%%%%%%%%%%%%
%       User definable parameters
%%%%%%%%%%%%%%%%%%%%%%%%%%%%%%%%%%%%%%%%%%%%%%%%%%%%%%%%%%%%%%%%%%
\newdimen\toobig		\newdimen\toobigfuzz
\newdimen\txttoscoreamt		
\newdimen\scoretonoteamt	
\newdimen\internoteamt
\newskip\notebaselineskip	\newdimen\noteleftmargin
\newskip\noterightskip		\newdimen\notewidth 
\newdimen\alwidth
\newdimen\contfirstcolwd	\newdimen\contblockleft
\newdimen\numgutter		
\newskip\interscoreskip
\newskip\intersinglineskip
\newskip\intercontskip
 
\def\workingcopy{
  \notewidth\hsize	% overall width of notes
  \notebaselineskip\basicleading	% space between baselines in notes
  \baselineskip2\basicleading
  \lineskip0pt \lineskiplimit0pt
  \contfirstcolwd1.36in
  \contblockleft.36in
  \noteleftmargin2em   % space the note block is indented from left margin
  \noterightskip2em plus2em % ragged effect for notes mimics score's raggedness
  \txttoscoreamt5pt	% space between txtline and score-block
  \scoretonoteamt5pt	% space between score-block and note
  \internoteamt6pt	% space between consecutive notes
  \interscoreskip2pc plus.5pc minus.5pc  % space between txt-score blocks
  \intersinglineskip1.5pc plus.5pc % space between `blocks' with only txt-line
  \intercontskip.1in % plus.1in % space between broken score block and its continuation
  \toobigfuzz1pc % ignore oversize alignments if they are less than this much 
                 % too big (c.f. \hfuzz and \vfuzz in Plain TeX)
  \numgutter=.15in
  \spacewidth1em \relax}

\def\verbnum#1.{{\bf#1.}\quad}
\def\BeginVerbBlock{
	\global\advance\donebits by 1
        \ifnum\donebits = 1 % no extra space before the first one
	  \else\vskip\interscoreskip
	\fi
        \vbox\bgroup\settxtfnt\parskip10pt\verbnum}
\def\EndVerbBlock{\egroup\filbreak\global\notect=0 \global\inscorefalse}

\let\Ell\ellipsis
\let\Eell\endellipsis
%\def\Slb{/\kern-2pt/}
\def\Slb{//}
\newdimen\pt \pt1pt

\def\zerowidth{0pt}
\def\score{\disabledots % halign swallows dots!
  \setupfonts \setupboxes
  \tabskip=0pt \parindent=0pt \parskip=0pt \hbadness=10000
  \vbadness=10000 \spaceskip=0pt \leftskip=0pt
  \hfuzz=\toobigfuzz % same as \toobigfuzz, but more widespread
  \workingcopy % default to a loose format
  \scorehook}
\def\endscore{\restoresize}

\def\scorehook{}

% This is put between parts of a split multispan
\def\Mfill{\hbox to1pt{\hfil}\hfill\hbox to1pt{\hfil}}
% This is put between adjacent square brackets
\def\Afill{\hbox to.25pt{\hfil}\hfill\hbox to.25pt{\hfil}}

% An edition will often have some commentary, so we provide a simple
% way of formatting that. If the user sets \showctyindexentriestrue
% the terms to be entered in the index are set in the left margin of the 
% commentary.
\newif\ifctylabelactive
\newskip\ctyinterskip
\def\ctydimens{\parskip2pt plus1pt \ctyinterskip6pt plus 1pt minus1pt \relax}

\newif\ifshowctyindexentries \showctyindexentriesfalse %true
\def\commentary{\ctydimens\commentaryhook}
\def\endcommentary{\indentnextpar}
\def\commentaryhook{}
\def\real@comma{,}
\newif\ifzeronextpar \zeronextpartrue
{\makeactive{,}%
\gdef\ad{\ifdim\lastskip=\ctyinterskip \else\vskip\ctyinterskip\fi
  \makeactive{,}\let,\allowbreak\real@ad}
\gdef\real@ad#1\on#2\par{\makeother{,}%
  \global\ctylabelactivetrue
  \nextpar{%
    \ifctylabelactive\def\next{\cty@formatlabel{#1}{#2}}\else\let\next\relax\fi
    \next
    \global\ctylabelactivefalse\ifzeronextpar\global\nextpar{}\fi}}%
\gdef\cty@formatlabel#1#2{%
  \ifshowctyindexentries\def\next{\cty@on@idx{#2}}\else\let\next\relax\fi
    \kern-\parindent
    \next \let,\real@comma
    \ctyformatlabel{#1}\ignorespaces}%
\gdef\cty@on@idx#1{%
  \llap{\setbox0=\hbox{\vtop{\nextpar{}\def,{\break\ignorespaces}%
    \parindent0pt \parfillskip0pt
    \baselineskip9pt\leftskip0pt plus1fill \rightskip0pt
    \relax\eightpoint\hsize1in\relax#1\endgraf}%
    \hbox to10pt{\hfil}}\ht0=0pt \dp0\ht0
    \box0}}}

\def\ctyformatlabel#1{\leavevmode
  \hboxtomin\indention{\bf\ignorespaces#1\unskip.\ }}

\def\clearctylabel{\global\ctylabelactivefalse}
\def\cref#1#2{#2}

\endinput

%%%%%%%%%%%%%%%%%%%%%%%%%%%%%%%%%%%%%%%%%%%%%%%%%%%%%%%%%

$Log: editions.tex,v $
Revision 0.21  1996/09/20 01:57:09  s
zero lineskip and lineskiplimit in default version of \workingcopy

Revision 0.20  1996/09/01 12:33:57  s
undo incorrect correction of \sum to \rm in \def\siglumfnt

Revision 0.19  1996/05/27 16:21:17  s
*** empty log message ***

% Revision 0.18  1996/02/20  01:31:41  s
% add support for \commentaryhook
%
% add \ifzeronextpar --- MUST MOVE THIS TO THE RIGHT PLACE!!!
%
% Revision 0.17  1996/02/17  23:32:07  s
% new definition of \formatsiglum
%
% Revision 0.16  1996/02/17  23:31:18  s
% new definition of \setupboxes using new font defs
%
% Revision 0.15  1995/11/18  11:24:08  s
% \cty@formatlabel now uses \hboxtomin so that the start of each numbered
% cty para is aligned with \indention
%
% Revision 0.14  1995/11/17  12:53:36  s
% fixed several stupid oversights in \ifshowctyindexentries change
%
% Revision 0.13  1995/11/17  12:04:16  s
% fixed printing of index entries for commentary
%
% Revision 0.12  1995/10/28  00:06:21  s
% added \scorehook to \def\score
%
% Revision 0.11  1995/10/13  10:13:38  s
% Use // for \Slb now that // is kerned properly in the fonts themselves
%
% Revision 0.10  1995/10/13  01:09:00  s
% added \def\scofnt to go with \txtfnt etc.
%
% Revision 0.9  1995/10/12  01:46:01  s
% fix boneheaded redefinition of \ad which was suppressing use of \ctyinterskip
%
% Revision 0.8  1995/10/10  01:34:16  s
% fixed all font calls from the old \csname<point><face>\endcsname
% notation to the new \csname<point>\endcsname\<face> notation.
%
% Revision 0.7  1995/09/21  01:04:01  s
% moresources magic
%
% Revision 0.6  1995/06/16  11:20:02  s
% slight reorganisation of \EndBlock
%
% added code to enable suppression of cty label formatting: headings now
% automatically \clearctylabel
%
% added \ctydimens called when a \commentary is begun
%
% added \ctyinterskip to control spacing between commentary blocks
%
% Revision 0.5  1995/03/01  16:11:48  s
% more undisciplined hacking for .edn files
%
% Revision 0.4  1995/02/24  13:25:14  s
% more fiddling with sources formatting
%
% Revision 0.3  1995/02/24  13:16:55  s
% More ad hoc hacking to make newstyle source-lists print OK
%
% Revision 0.2  1994/12/13  02:29:54  s
% Fixed \restoresize by moving \rm inside of globaldefs1
%
% Revision 0.1  1994/12/12  03:56:05  s
% initial checkin
%
