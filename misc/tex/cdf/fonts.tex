% FONTLOAD
% Generic Font Loading On Demand (FLOD) package for TeX
%
% The idea is to set things up so that \rm selects the current size of
% the roman font of the current face. Similarly, \seven\rm selects
% the sevenpoint roman font of the current face. \CM\seven\rm
% selects the sevenpoint roman face of the CM fonts.
%
% In all cases the prior loading of the font is tested for, and
% the font is only loaded if not found.
%

\newtoks\FLODtextintcoll % where we keep the internal representation of a text FLODcoll
%	\FLODtextintcoll{DACMR}%initialize for FLODpushcol
\newtoks\FLODmathintcoll % where we keep the internal representation of the math coll
%	\FLODmathintcoll{CMM}%initialize for FLODpushcol
\newtoks\FLODintcol@ % the current collection for font constructors
\newtoks\FLODintsize % where we keep the internal representation of a FLODsize
\newtoks\FLODextsize % where we keep the external representation of a FLODsize
\newtoks\FLODintsiz@ % the internal form of the next font to load
\newtoks\FLODextsiz@ % the external form of the next font to load
\newtoks\FLODintface % where we keep the internal representation of a FLODface

\FLODintsize{ten}\FLODextsize{10}

% We begin by defining a means of initializing a control sequence so that
% it loads a file of macros the first time it's called
%
% #1 = collection name, e.g. DACM
% #2 = definition module, e.g. fnt-dacm
\def\FLODinit#1#2{\expandafter\gdef\csname#1\endcsname{%
  \FLODshow{Received request to initialize collection #1 from #2}%
  \ifFLODloadOK\def\next{\FLODreallyload{#1}{#2}}%
  \else\def\next{\FLODloaderror{#1}}\fi
  \next}}

\def\FLODreallyload#1#2{%
  \FLODshow{Loading font collection #1 ...}%
  \undefine{\csname#1\endcsname}%		first undefine the collection name
  \expandafter\def\csname#1hook\endcsname{\let\currface\FLODcurrface}%
						% define a default hook
  \macrofile{lib/#2}%				now load the font macros for it
  \expandafter
  \if\csname#1\endcsname\undefd %	and check that it got defined
    \errhelp{The font definition module `#2' did not contain a definition of^^J%
            the font collection `#1'.}%
    \errmessage{Font collection `#1' undefined}%
  \else
    \csname#1\endcsname
  \fi}
\def\FLODloaderror#1{%
  \errhelp{You must declare fonts with `usefontcollection' before trying to use them.}%
  \errmessage{Attempt to use undeclared font collection `#1'}}

% \FLODcoll and \FLODmathcoll:
% ============================
%
%   #1 = internal collection name, e.g. DACMR
%   #2 is presently ignored
%
% After this, \DACMR is redefined to set the internal collection
% variable to the DACMR.  Before the first invocation, the cs is
% defined in fontlist.tex to load the macros for that font.
%
% The hook macro is called before using \currface because that allows
% collections to do face overloading and supply their own \currface 
% switch.
\def\FLODcoll#1{\expandafter\gdef\csname#1\endcsname{%
	\FLODtextintcoll{#1}%
	\csname#1hook\endcsname
	\currface}%
  \FLODtextintcoll{#1}}
\def\FLODmathcoll#1#2{\expandafter\gdef\csname#1\endcsname{%
	\FLODmathintcoll{#1}\csname#1hook\endcsname}%
	\FLODmathintcoll{#1}}
% This is like FLODcoll, but it doesn't take effect until a face setting command
%\def\FLODcollnocurrface#1#2{%
%  \expandafter\gdef\csname#1\endcsname{\FLODtextintcoll{#1}\csname#1hook\endcsname}%
%	\FLODtextintcoll{#1}}

% \FLODface:
% ==============
%
% This defines how to look up the external face from the internal collection
% and face.
% #1 = internal collection, e.g. DACMR
% #2 = face name, e.g. rm
% #3 = external font name, e.g., cmrr4
% after this one can say \csname DACMRrm!ext\endcsname to get cmrr4
\def\FLODface#1#2#3{\expandafter\gdef\csname#1#2!ext\endcsname{#3}}

% \FLODsetsizes:
% ==============
%
% A font definition file contains commands like this:
%	 \FLODsetsizes{DACMR}{10}{8}{6}{12}
% #1 is collection
% #2 is text size
% #3 is script size
% #4 is scriptscript
% #5 is baseline (single-spaced, no fuzz)
% e.g. \FLODsetsizes{DACMR}{10}{7}{5}{12} defines
%		\DACMR10t as 10
%		\DACMR10s as 7
%		\DACMR10ss as 5
%		\DACMR10b as 12
\def\FLODsetsizes#1#2#3#4#5{%
  \expandafter\gdef\csname#1#2t\endcsname{#2}%
  \expandafter\gdef\csname#1#2s\endcsname{#3}%
  \expandafter\gdef\csname#1#2ss\endcsname{#4}%
  \expandafter\gdef\csname#1#2b\endcsname{#5}}

% \FLODsetfamily
% ==============
%
% Because commands like \rm need to reset the components of families
% we need to define a means of associating the face commands with a family
% ID and an indication of which sizes the face is available for, i.e.,
% textsize, scriptsize and scriptscriptsize. By arranging that this command
% be optional we allow non-family fonts (like the IPA symbols) simply to omit
% it. Note that a face can specify that two or all three of the family
% members share the same size. The size is derived from the FLODsetsizes
% command for the relevant collection and base-size.
%
% #1 is collection
% #2 is face
% #3 is text font to use*
% #4 is script font to use*
% #5 is scriptscript font to use*
% #6 is family ID to select when face command is executed; 
%    -1 for no change
% * these values are indexed into text/script/scriptscript as defined 
%   by a FLODsetsizes command
%
% e.g. \FLODsetfamily{CM}{rm}{0}{1}{2}{0} will define:
%		\CMrmfam{0}
%		\CMrmsizes{012}
\def\FLODsetfamily#1#2#3#4#5#6{\count255=#6%
  \expandafter\xdef\csname#1#2fam\endcsname{\the\count255}%
  \expandafter\xdef\csname#1#2sizes\endcsname{#3#4#5}}

%%%%%%%%%%%%%%%%%%%%%%%%%%%%%%%%%%%%%%%%%%%%%%%%%%%%%%%%%%%
%
% Now we come to the interesting stuff.  Here are the macros
% that actually implement font loading on demand.
%

% check to see if the fonts for this family have been loaded.  If so,
% just use \FLODsetonefamily.  If not, load all the fonts needed.
%
% When this is called, the collection, size and face have all been
% set.  #1 is the face abbreviation, e.g. `rm'.
\def\FLODtextloadorsetonefamily#1{\FLODintcol@\FLODtextintcoll
  \FLODloadorsetonefamily{#1}}
\def\FLODmathloadorsetonefamily#1{\FLODintcol@\FLODmathintcoll
  \FLODloadorsetonefamily{#1}}

\def\FLODloadorsetonefamily#1{%
  \FLODintface{#1}%
  \ifcsname{\the\FLODintcol@\the\FLODintface\the\FLODextsize f}%
    \relax\FLODloadonefamily
  \FLODsetonefamily{#1}}

\def\FLODloadonefamily{%
  \ifcsname{\the\FLODintcol@\the\FLODextsize t}%
    \FLODreally@loadfamily\FLODcant@loadfamily}

\def\FLODcant@loadfamily{%
  \errhelp{%
There is no `FLODsetsizes' command for current point size^^J%
in the current font module file.  That means that I don't know how^^J%
to set the text, script and scriptscript fonts for the various families.}%
  \errmessage{`FLODsetsize' not called for \the\FLODintcol@ \the\FLODextsize}}

\def\FLODreally@loadfamily{%
  \expandafter\gdef\csname\the\FLODintcol@\the\FLODintface\the\FLODextsize f\endcsname{t}%
  \edef\next{\csname\the\FLODintcol@\the\FLODintface sizes\endcsname}%
  \expandafter\FLODloadbyfamsizes\next}

% set \@tsize, \@ssize and \@sssize to the first, second or third order 
% size of <COLL><FACE><SIZE> depending on the value of the definition of
% <COLL><FACE>sizes. Arg 1 is for text, arg 2 is for script and arg 3
% is for sscript.
\def\FLODloadbyfamsizes#1#2#3{%
  \FLODloadone{\ifcase#1 t\or s\or ss\fi}t%
  \FLODloadone{\ifcase#2 t\or s\or ss\fi}s%
  \FLODloadone{\ifcase#3 t\or s\or ss\fi}{ss}}

\def\FLODloadone#1#2{%
  \edef\FLOD@nextsize{\csname\the\FLODintcol@\the\FLODextsize#1\endcsname}%
%  \message{FLOD@nextsize is \FLOD@nextsize}
  \expandafter\FLODextsiz@\expandafter\expandafter{\FLOD@nextsize}%
  \edef\FLOD@nextsize{\csname FLOD\the\FLODextsiz@\endcsname}%
  \expandafter\FLODintsiz@\expandafter\expandafter{\FLOD@nextsize}%
  \FLODloadfont
  \expandafter
    \xdef
      \csname\the\FLODintcol@\the\FLODintface\the\FLODextsize#2\endcsname{\lastfontloaded}}

\def\FLODloadfont{%
  \FLODshow{Received request to load \the\FLODintcol@\the\FLODintface\the\FLODintsiz@}%
  \isfontcs{\the\FLODintcol@\the\FLODintface!\the\FLODextsiz@}%
  \iffont
    \deflastfontloaded
    \def\next{\FLODshow{\lastfontloaded\space already loaded}}%
  \else
    \def\next{\FLODfindfont}%
  \fi
  \next}

% Look for a font which has not yet been loaded
\def\FLODfindfont{%
  \edef\tryfont{"\csname\the\FLODintcol@\the\FLODintface!ext\endcsname" %
  at \csname FLOD\the\FLODintsiz@\endcsname pt}%
  \FLODshow{requested: \tryfont}%
  \FLODreallyloadfont{\tryfont}}

% Load the argument font to the cs 
% `\the\FLODintcoll\the\FLODintface\the\FLODintsize' 
% and also define the macro that tracks what has been loaded
% for future testing of the font cs.
\def\FLODreallyloadfont#1{\FLODshow{loading #1}%
  \global\expandafter\font\csname\the\FLODintcol@\the\FLODintface\the\FLODintsiz@\endcsname=#1%
  \deflastfontloaded\defnextfonttouse
  \expandafter\gdef\csname\the\FLODintcol@\the\FLODintface!\the\FLODextsiz@\endcsname{!OK!}}

% Same as above, but used when no TFM could be found; sets the font
% to nullfont, but pretends the font has been loaded so it doesn't get
% looked for needlessly in the future.
\def\FLODrelaxfont#1{\FLODshow{relaxing #1}%
  \expandafter\xdef\csname\the\FLODintcol@\the\FLODintface\the\FLODintsiz@\endcsname{\nullfont}%
  \deflastfontloaded\defnextfonttouse
  \expandafter\gdef\csname\the\FLODintcol@\the\FLODintface!\the\FLODextsiz@\endcsname{!OK!}}

\def\deflastfontloaded{%
  \xdef\lastfontloaded{\csname\the\FLODintcol@\the\FLODintface\the\FLODintsiz@\endcsname}}
\def\defnextfonttouse{%
  \xdef\nextfonttouse{\csname\the\FLODintcol@\the\FLODintface\the\FLODintsiz@\endcsname}}

%%%%%%%%%%%% set up the family after the fonts have been loaded

\def\FLODtextsetonefamily#1{\FLODintcol@\FLODtextintcoll\FLODsetonefamily{#1}}
\def\FLODmathsetonefamily#1{\FLODintcol@\FLODmathintcoll\FLODsetonefamily{#1}}

% This macro uses the previously defined cs's of the form
%	  \<COLL><FACE><SIZE>{t|s|ss}
% to set the text/script/scriptscript fonts quickly. SIZE is in digits.
% It takes one argument, the font abbreviation, e.g. `rm'.
\def\FLODsetonefamily#1{%
  \expandafter
    \textfont
      \csname#1fam\endcsname=\csname\the\FLODintcol@#1\the\FLODextsize t\endcsname
  \expandafter
    \scriptfont
      \csname#1fam\endcsname=\csname\the\FLODintcol@#1\the\FLODextsize s\endcsname
  \expandafter
    \scriptscriptfont
      \csname#1fam\endcsname=\csname\the\FLODintcol@#1\the\FLODextsize ss\endcsname}

%%%%%%%%%% A convenient way of preloading all fonts for all families

\def\FLODsetfamilies{%
  \ifcsname{\the\FLODintcoll\the\FLODextsize f}\FLODreally@setfamilies\FLODcant@setfamilies}

\def\FLODcant@setfamilies{%
  \errhelp={You must use `usefontsize' for each size you want to use^^Jin
each font collection before using xxxpoint macros.}%
  \errmessage{`usefontsize' not called for \the\FLODintcoll \the\FLODintsize}}

% This macro is used for setting the families up after the fonts
% have already been loaded by a usefontsize command.  It uses the 
% lists of fonts defined in FLODfacelist and FLODmathfacelist to 
% apply FLODsetonefamily to each family.
\def\FLODreally@setfamilies{%
  \let\FLODdo\FLODtextsetonefamily\FLODtextfacelist
  \let\FLODdo\FLODmathsetonefamily\FLODmathfacelist}

%%%%%%%%%%%%%%%%%%%%%%%%%%%%%%%%%%%%%%%%%%%%%%%%%%%%%%%%%%%

% \FLODsize:
%
%   #1 = size in words, e.g. `seven'
%   #2 = size in digits, e.g. `7'
%
% The controls sequences \seven and \sevenpoint defined in this way
% do not change the current face or collection.
\def\FLODsize#1#2{%
  \expandafter\gdef\csname FLOD#1\endcsname{#2}% look up 10 from ten
  \expandafter\gdef\csname FLOD#2\endcsname{#1}% look up ten from 10
  \expandafter\gdef\csname#1\endcsname{\FLODintsize{#1}\FLODextsize{#2}%
				       \currface\FLODmathdirty}%
  \expandafter\gdef\csname#1point\endcsname{\FLODintsize{#1}\FLODextsize{#2}%
					    \currface\FLODmathdirty}}
\def\FLODmathdirty{}%%%\def\everymathextra{\FLODmathclean}}
\def\FLODmathclean{}%%%%
%  \let\FLODdo\FLODmathloadorsetonefamily\FLODmathfacelist
%  \def\everymathextra{}}

% the way the DA format uses math will need some fixing for real math;
% perhaps a macro such as \realmath which temporarily omits the
% \fam\currfam
\everymath{\fam\currfam\everymathextra}

\def\FLODfacelist{%
  \FLODdo{rm}\FLODdo{it}\FLODdo{bf}\FLODdo{bi}%
  \FLODdo{su}\FLODdo{sub}\FLODdo{ak}\FLODdo{akb}}

\def\FLODmathfacelist{\FLODdo{mi}\FLODdo{sy}\FLODdo{ex}%
  \FLODdo{mib}\FLODdo{syb}}

% This is used to set up \rm and friends
\def\FLODuseallfamilies{%
  \let\FLODdo\FLODtextuseonefam\FLODfacelist
  \let\FLODdo\FLODmathuseonefam\FLODmathfacelist}

\def\FLODtextuseonefam#1{%
  \expandafter\gdef\csname#1\endcsname{%
%     \@protect{\FLODsetonefamily{#1}}{\FLODloadorsetonefamily{#1}}
     % by not protecting this call we force an error if a font is used
     % in a \mark that hasn't been used outside of the \mark.  The FLOD
     % macros can't autoload within a \mark, but at least we can complain and
     % then the user can just use the font spuriously in a safe place to 
     % prepare it for later.
     \FLODtextloadorsetonefamily{#1}%
     \fam\csname#1fam\endcsname\currfam\fam % reset the family and \currfam
     \csname\the\FLODintcol@#1\the\FLODextsize t\endcsname % set curr font
     \csname#1hook\endcsname}} % call the hook function for the face

% like the above, but for math fonts
\def\FLODmathuseonefam#1{%
  \expandafter\gdef\csname#1\endcsname{%
     \FLODmathloadorsetonefamily{#1}%
     \fam\csname#1fam\endcsname\currfam\fam % reset the family and \currfam
     \csname\the\FLODintcol@#1\the\FLODextsize t\endcsname}} % set curr font

\FLODuseallfamilies % set up \rm and friends

% we need to know if a \cs is a currently defined font selector
% so we provide:
%
%   \isfont: test whether a control sequence is a font selector
%   \isfontcs: the same, but argument is encapsulated in \csname...\endcsname
%
% After either call the conditional \iffont is true if the csname
% selects a font, false otherwise.
%
% We do this by defining a control sequence for each font that expands
% to the string !OK! and simply comparing it to the constant \known@font.
%
% To avoid the possibility that a user macro will both have this definition
% and get tested by this macro (both pretty small), we put '!' in the csname
% also. If a user defines a macro with '!' in the name, gives it the value !OK!
% *and* tests it with this macro, let's just trust that they know what they're
% doing.
\def\known@font{!OK!} \newif\iffont
\def\isfont#1{\ifx#1\known@font\fonttrue\else\fontfalse\fi}
\def\isfontcs#1{\expandafter\isfont\csname#1\endcsname}

% If \tracingfonts=1 we show every fontload operation on screen and
% log. If \tracingfonts=2 we show operations on log only. (If 0, no show.)
\newcount\tracingfonts \tracingfonts=2
\def\FLODshow#1{\ifcase\tracingfonts\or\wstdout{Font: #1}\or\wlog{Font: #1}\fi}


% We predefine the font sizes, which are generic, and the basic family
% names, which are limited and so must be conventionalized across
% modules. Everything else is defined in load-on-demand macro modules.
%
% make 'fam' naming consistent
\count18=-1 % careful now; must define the first 4 like Plain...
\newfam\rmfam
\newfam\mifam
\newfam\syfam
\newfam\exfam
\newfam\itfam
\newfam\bffam
\newfam\bifam
\newfam\mibfam
\newfam\sybfam
\newfam\sufam
\newfam\subfam
\newfam\akfam
\newfam\akbfam
\newfam\Xfam % a bizarre, rotating family used as a wildcard
	     % actually, I'm not sure what I'm going to do with this yet

% this macro must be defined with some care. The actual numbers to 
% which the various \fam's correspond need not be known, but the 
% order of selection must be the same as the order of calls to 
% \newfam.
% 
% Would be neat to rewrite the newfam and currface stuff as a macro
% which defined everything based on a list of family names.
\newcount\currfam \currfam-1
\def\savef@m{\currfam\the\fam\relax}
\def\FLODcurrface{%
  \ifcase\currfam
    \rm\or
    \mi\or % math italic = 1
    \sy\or % math symbol = 2
    \ex\or % math extended = 3
    \it\or % = 4
    \bf\or % = 5
    \bi\or % = 6
    \mib\or % = 7
    \syb\or % = 8
    \su\or % = 9
    \sub\or % = 10
    \ak\or % = 11
    \akb\or % = 12
    \Xface % = 13
  \fi}

%Initialize \currface
\let\currface\FLODcurrface

% The alias \setfonts for \currface allows users to say \seven\setfonts
% and \CM\setfonts, inheriting face and/or size from the environment.
\def\setfonts{\currface}

\FLODsize{three}{3}
\FLODsize{four}{4}
\FLODsize{five}{5}
\FLODsize{six}{6}
\FLODsize{seven}{7}
\FLODsize{eight}{8}
\FLODsize{nine}{9}
\FLODsize{ten}{10}
\FLODsize{eleven}{11}
\FLODsize{twelve}{12}
\FLODsize{thirteen}{13}
\FLODsize{fourteen}{14}
\FLODsize{fifteen}{15}
\FLODsize{sixteen}{16}
\FLODsize{seventeen}{17}
\FLODsize{eighteen}{18}
\FLODsize{nineteen}{19}
\FLODsize{twenty}{20}
\FLODsize{twentyone}{21}
\FLODsize{twentytwo}{22}
\FLODsize{twentythree}{23}
\FLODsize{twentyfour}{24}
\FLODsize{twentyfive}{25}

% Now we define a few defaults for handling fonts.
% We make it easy to get everything in bold face if so desired
\def\@emfonts{\let\alo\bf\let\sum\sub\let\sumup\sub\let\akk\akb\def\rot{\rotbold}}
\def\@rmfonts{\let\alo\rm\let\sum\su\let\sumup\su\let\akk\ak\def\rot{\rotrom}}
\def\bem{\@emfonts\bf} \def\em{\@rmfonts\it} \def\unem{\@rmfonts\rm}
% one could get bold superscripts as well by saying \let\sumup\sub
\def\boldsum{\let\sum\sub\let\sumup\su\def\rot{\rotbold}}

% for compatibility with Plain TeX
\def\oldstyle{\fam\@ne\mi} \def\mit{\mi}

% Because it's so problemmatic to allow the font collection macros
% to be loaded within a group, we define the following and require
% that font collections to be used be declared thusly.
%
%   \usefontcollection{\DAComputerModern}
%
% It's really not much more than syntactic sugar, but it forces 
% users to do font preloading.
\newif\ifFLODloadOK
\def\usefontcollection#1{\FLODreallyusecoll{#1}}
\def\FLODreallyusecoll#1{%
  \FLODpushcoll\FLODloadOKtrue
  #1
  \FLODloadOKfalse\FLODpopcoll}
\def\FLODusecollerror#1{%
  \errhelp{You are trying to load a font that is not defined in fontlist.tex}%
  \errmessage{Font \string#1 has not been initialized}}

%%% FIXME !!!???
%%% There may be other things that should be pushed and popped around
%%% a \usefontcollection command: keep an eye on this!
\newcount\FLOD@restoreface
\newcount\FLOD@pushedfam    \FLOD@pushedfam0
\newtoks\FLODtextpushedcoll
\newtoks\FLODmathpushedcoll
\def\FLODpushcoll{
  \FLOD@pushedfam\currfam
  \let\FLOD@pushedsaveface\currface
  \FLODtextpushedcoll\FLODtextintcoll
  \FLODmathpushedcoll\FLODmathintcoll}
\def\FLODpopcoll{
  \currfam\FLOD@pushedfam
  \let\currface\FLOD@pushedsaveface
  \FLODtextintcoll\FLODtextpushedcoll
  \FLODmathintcoll\FLODmathpushedcoll
  \ifnum\currfam=-1 \else\currface\fi}

%%%%%%%%%%%%%%%%%%%%%%%%%%%%%%%%%%%%%%%%%%%%%%%%%%%%%%%%%%%%%%%%%%%%%%%%%%%%%%%%%%
%% FONT; Generic font loader which works only with XeTeX.
% (The old font loader has been dropped and only XeTeX is supported by the CDF
% macros)

% e.g., \FLODset{Ungkam}{UNG}{Ungkam}
% where 
%  #1 is the control sequence to be defined; 
%  #2 is the internal code; 
%  #3 is the registered font name

\def\FLODset#1#2#3{\FLODshow{Loading #1}%
  \FLODcoll{#2}%
  \FLODface{#2}{rm}{#3}%
  \FLODface{#2}{it}{#3/I}%
  \FLODface{#2}{bf}{#3/B}%
  \FLODface{#2}{bi}{#3/BI}%
%\FLODface{#2}{su}{pnls4}
%\FLODface{#2}{ak}{pnla4}
%\FLODface{#2}{sub}{pnlsb4}
%\FLODface{#2}{akb}{pnlab4}
  \FLODsetfamily{#2}{rm}011\rmfam
  \FLODsetfamily{#2}{it}011\itfam
  \FLODsetfamily{#2}{bf}011\bffam
  \FLODsetfamily{#2}{bi}011\bifam
%\FLODsetfamily{#2}{su}011\sufam
%\FLODsetfamily{#2}{ak}011\akfam
%\FLODsetfamily{#2}{sub}011\subfam
%\FLODsetfamily{#2}{akb}011\akbfam
  \FLODsetsizes{#2}{5}{5}{5}{7}%
  \FLODsetsizes{#2}{6}{6}{5}{8}%
  \FLODsetsizes{#2}{7}{6}{5}{9}%
  \FLODsetsizes{#2}{8}{6}{5}{10}%
  \FLODsetsizes{#2}{9}{7}{5}{11}%
  \FLODsetsizes{#2}{10}{7}{5}{13}%
  \FLODsetsizes{#2}{11}{8}{6}{14}%
  \FLODsetsizes{#2}{12}{9}{7}{15}%
  \FLODsetsizes{#2}{14}{11}{9}{18}%
  \FLODsetsizes{#2}{17}{14}{10}{21}%
  \FLODsetsizes{#2}{20}{14}{10}{24}%
  \FLODsetsizes{#2}{25}{17}{12}{29}%
}

%%%%%%%%%%%%%%%%%%%%%%%%%%%%%%%%%%%%%%%%%%%%%%%%%%%%%%%%%%%%%%%%%%%%%%%%%%%%%%%%%%
%% FONTSETS
% This module defines a simple way of establishing what the meanings
% of \Serifed, \Sansserif and \Fixedwidth are. The idea is to use
% these general font styles as an API and thus avoid even font names
% in user macros, except at the very beginning of a document.

% This is how we set up the meanings of \Serifed, \Sansserif and
% \Fixedwidth
%
%  #1 is the font collection to use for serifed text
%  #2 is the font collection to use for sansserif text
%  #3 is the font collection to use for fixed width text
%
% The apparently odd order of the usefontcollection commands is
% intentional: if no font collection has been set yet, the serifed
% one will be the persistent one.
\def\definefontset#1#2#3{%
  \def\Serifed{#1}\def\Sansserif{#2}\let\Sanserif\Sansserif
  \def\Fixedwidth{#3}%
  \usefontcollection#2\usefontcollection#3\usefontcollection#1\relax}

% shorter names for Sanserif and Fixedwidth
\def\Sans{\Sanserif}\def\Fw{\Fixedwidth}\def\tt{\Fixedwidth}

% This is how to set text in the right superscript size relative to the current size
\def\scriptsize{%
  \edef\next{\csname FLOD\csname\the\FLODtextintcoll\the\FLODextsize s\endcsname\endcsname}%
  \csname\next\endcsname}

\def\logosize{%
  \edef\next{\csname FLOD\csname\the\FLODtextintcoll\the\FLODextsize s\endcsname\endcsname}%
  \csname\next\endcsname}

%\def\textsize{%
%  \edef\next{\csname FLOD\csname\the\FLODtextintcoll\the\FLODextsize \endcsname\endcsname}%
%  \csname\next\endcsname}


%%%%%%%%%%%%%%%%%%%%%%%%%%%%%%%%%%%%%%%%%%%%%%%%%%%%%%%%%%%%%%%%%%%%%%%%%%%%%%%%%%
%% Default Fontsets
\global\tracingfonts=2
\FLODset{CuneiformNA}{CunNA}{CuneiformNA}
\FLODset{UngkamBookBasic}{UBB}{Ungkam Book Basic}
\FLODset{DejaVuSans}{DVS}{DejaVu Sans}
\FLODset{DejaVuMono}{DVM}{DejaVu Mono}
\definefontset{\UBB}{\DVS}{\DVM}
%\UBB\ten\rm

\endinput
