% This file is part of the Digital Assyriologist.  Copyright
% (c) Steve Tinney, 1994, 1995.  It is distributed under the
% Gnu General Public License as specified in /da/doc/COPYING.
%
% $Id: env.tex,v 0.7 2001/10/16 16:55:26 s Exp s $

% environment definitions.
% supported environments and their effects are:
% shapes
%	centerpars	centerlines
%	leftpars	leftlines
%	rightpars	rightlines
%       hangpars
% modes
%	address
%	#column % #=number, invoke multiple column section
%       comment
%       commentary
%       composite
%       score
%       sources
%       translation
%       verbatim
% displays
%       center
%       display
%	poetry
%	quote
%       cquote
%       tquote
% lists:
%	description
%	enumerate
%	itemize
%	outline
% first set up an alternative for \end, and redefine \bye
\let\End\end
\outer\gdef\bye{\par\vfill\supereject\End}

\newif\ifdisplay
\newif\ifalwaysindent \alwaysindentfalse

%% here's the basic control structure
\def\begin#1{%
  \indentnextpar
  \begingroup
  \def\@currenv{#1}%
  \csname#1\endcsname}% don't use any \ignorespaces in case a \begin{xx}
                          % takes an argument
\def\end#1{%
  \def\next{#1}%
  \ifx\@currenv\next%
  \else%
    \errhelp={@begin...@end groups must not be interwoven: you just ended a^^J%
	group with an @end command that was different from the one that^^J%
	began it. (I'll perform the requested @end anyway, and we'll see^^J%
	what happens.)}%
    \edef\next{Mismatched \noexpand @begin...\noexpand @end (expected `\@currenv', found `#1')}%
    \expandafter\errmessage\expandafter\expandafter{\next}%
  \fi%
  \csname end#1\endcsname%
  \endgroup\ignorespaces}

% The simplest environments are those that set up particular shapes:
\def\rightpars{%
  \parindent0pt \advance\leftskip2em plus1fil\relax
  \dimen0=\rightskip \rightskip=\dimen0 } % coerce \rightskip into losing fil
\def\leftpars{%
  \parindent0pt \advance\rightskip2em plus1fil\relax \hyphenpenalty10000 
  \exhyphenpenalty10000 \dimen0=\leftskip \leftskip=\dimen0 } % coerce \leftskip into losing fil
\def\centerpars{%
  \advance\leftskip=0pt plus1fil \advance\rightskip0pt plus1fil \parindent=0pt}
\let\centrepars\centerpars
\def\hangpars{\parindent0pt \everypar{\the\nextpar\hangindent\indention}}
\let\endleftpars\par
\let\endrightpars\par
\let\endcentrepars\par
\let\endhangpars\par

{\makeactive{\^^M}%
  \gdef\rl@ines{\parskip=0pt \par\noindent \parfillskip=0pt %
  \leftskip0pt plus1fil \everypar{} \let^^M\par}%
  \gdef\rightlines{\makeactive{\^^M}\def^^M{\rl@ines}}}
\def\leftlines{\parskip=0pt \par\noindent \parindent=\parskip \obeylines}
\def\centerlines{\parskip=0pt \par\noindent \parfillskip\parskip 
  \parindent\parskip \leftskip=0pt plus1fil \rightskip\leftskip \obeylines}
\let\centrelines\centerlines
\let\endrightlines\relax
\let\endleftlines\relax
\let\endcenterlines\relax

\def\pagegroup{\endgraf\vbox\bgroup}
\def\endpagegroup{\egroup\noindent}

\input editions
\input composit

%                                   address environment: 
%                                   set the address aligned along the left edge
%                                   of the longest line
%                                   like this
% We do this by gathering everything in a vbox and outputting it when the
% \end{address} is reached
\def\append#1{\global\setbox1=\vtop{\box1\box#1}}
{\catcode`\^^M=\active%
\gdef\obeyshortlines{\catcode`\^^M=\active\setbox0=\hbox\bgroup}%
\gdef^^M{\strut\egroup\append0\obeyshortlines}}
\def\address{\global\setbox1=\vbox{}\obeyshortlines}
\def\endaddress{\egroup\append0\rightline{\box1}\bigskip}

% The comment macros are based on Comment.sty version 1.0 2 November 1990
% originally written by eijkhout@csrd.uiuc.edu
%
% Basic approach: take every line in verbatim mode as macro argument, if
% it doesn't exactly match the end text discard it. This means no extra
% spaces on the @end{comment} line. For verbatim text output the line
% in an hbox. 
\newif\ifshowingcomments \showingcommentsfalse
\def\comment{\ifshowingcomments\cvd@splay\fi\verbatimformat\c@mment}
\def\verbatim{\cvd@splay\verbatimformat\v@rbatim}
\def\cvd@splay{\@begdisplay \parskip0pt \parindent0pt\relax}
{\catcode`\^^M=\active %
 \gdef\c@mment#1^^M{\def\test{#1}%
 \ifx\test\DAendcommenttest\def\next{\DAformat\end{comment}}%
 \else\def\next{\ifshowingcomments\vfl@ne{#1}\fi\c@mment}\fi%
 \next}%
 \gdef\v@rbatim#1^^M{\def\test{#1}%
 \ifx\test\DAendverbatimtest\def\next{\DAformat\end{verbatim}}%
 \else\def\next{\vbl@ne{#1}\allowbreak\v@rbatim}\fi%
 \next}}
\def\vbl@ne#1{\noindent\verbatimfont#1\endgraf}
\def\verbatimfont{\Fixedwidth\nine\rm}
% define an @end{comment} in which all chars have catcode 12
\begingroup
  \catcode`@12 \catcode`{12 \catcode`}12 \catcode`(1 \catcode`)2
  \xdef\DAendcommenttest(@end{comment})
  \xdef\DAendverbatimtest(@end{verbatim})
\endgroup
\def\endcomment{\ifshowingcomments\@enddisplay\fi}
\def\endverbatim{\@enddisplay}

% list environments
% allocate variables
\newdimen\@ndentamt     \newdimen\nostuffindent
\newcount\@ndentlev     \@ndentlev=0 % global indentation level, for lists only
\newcount\en@mlev       \en@mlev=0 % currently nested enumerates
\newcount\@utlinelev    \@utlinelev=0 % current nesting of outline
\newcount\@listnestlev  \@listnestlev0 % current global list depth
\newcount\en@mi         \newcount\@utlinei              \newcount\@utlinevi
\newcount\en@mii        \newcount\@utlineii             \newcount\@utlinevii
\newcount\en@miii       \newcount\@utlineiii            \newcount\@utlineviii
\newcount\en@miv        \newcount\@utlineiv             \newcount\@utlineix
\newcount\en@mv         \newcount\@utlinev              \newcount\@utlinex
\newtoks\en@mtoks % used by both outline and enum for static part of label
\newdimen\en@mlabwd
\newtoks\en@mlabtoks
\newskip\envparskip  \envparskip=2pt plus.1pt
\newskip\subabovedisplayskip \newskip\subbelowdisplayskip

% now a few convenient shorthands, etc
\def\@Above{\removelastskip\nobreak\vskip\abovedisplayskip \nointerlineskip}
\def\@Below{\removelastskip\vskip\belowdisplayskip\relax}
\def\sub@Above{\removelastskip\nobreak\vskip\subabovedisplayskip \nointerlineskip}
\def\sub@Below{\removelastskip\vskip\subbelowdisplayskip\relax}
\def\abovedisplay{\@Above}
\def\belowdisplay{\@Below}
\def\abovesubdisplay{\sub@Above}
\def\belowsubdisplay{\sub@Below}
\def\listbegin{%
  \subabovedisplayskip.5\abovedisplayskip
  \subbelowdisplayskip.5\belowdisplayskip
  \ifvmode\else\unskip\fi
  \par\listspacing
  \ifnum\@listnestlev=0 \@Above\advance\rightskip\@ndentamt\else\sub@Above\fi
  \parindent0pt \parskip\envparskip \@ndentamt\indention
  \advance\leftskip\@ndentamt % locally set left margin (reset by \seten@mlabwd)
  \advance\@ndentlev1 % locally bump indent level for enumerate etc.
  \advance\@listnestlev1 } % locally bump global nest level
\def\listend{\par 
  \advance\@ndentlev-1 \advance\@listnestlev-1
  \ifnum\@listnestlev=0 \@Below\else\sub@Below\fi
  \nointerlineskip }
\def\setindentamount#1{\@ndentamt#1}
\def\setindentlevel#1{\@ndentlev#1\relax
  \leftskip\@ndentlev\@ndentamt} % locally reset left margin
% If more than one para is needed in an itemize or enumerate one
% can say \nostuff before each para to skip the stuff that everypar
% would otherwise put in there.
\def\nostuff#1\par{{\everypar{\the\nextpar}\noindent\hangindent\@ndentamt
  \kern\@ndentamt \kern\nostuffindent
  \ignorespaces#1\endgraf}}

% enumeration is defined for 1, 1.1, 1.1.1, 1.1.1.1, style 
% Only five levels are supported, as the outline environment is 
% intended for use in more complex situations (see below).
\def\seten@mtoks{%
  \global\en@mtoks={%
  \count255=1%
  \loop 
    \ifnum\count255 > 1 .\number%
      \ifcase\count255 \or\en@mi\or\en@mii\or\en@miii\or\en@miv\or\en@mv\fi
    \else \number\en@mi 
    \fi
    \advance\count255by1 %
    \ifnum\count255<\@ndentlev 
  \repeat .}}
\def\seten@mlab{\global\advance\activecount1
  \global\en@mlabtoks={%
    \ifnum\@ndentlev>1 \the\en@mtoks\fi % static part of label
    \number\activecount .}} % final number and period
\def\enumeverypar{\seten@mlab % prepare the actual label for this para
  \leavevmode\hbox to\en@mlabwd{\rm\the\en@mlabtoks\hfil}\hangindent\en@mlabwd
  \ignorespaces}
\def\seten@mlabwd{\setbox0=\hbox{\rm00.} \en@mlabwd\wd0 
  \multiply\en@mlabwd\@ndentlev
  \advance\leftskip-\@ndentamt % reset leftskip before adding the whitespace
  \advance\leftskip\en@mlabwd  % to the \en@mlabwd
  \advance\leftskip-\wd0       % need to align below last level's number width
  \ifnum\@ndentlev>1           % and last level's space if this is not the 
    \advance\leftskip.5em \fi  % first level
  \advance\en@mlabwd.5em\relax}
\def\enumerate{%
  \ifnum\@listnestlev=0 \@ndentlev=0 \en@mlabwd0pt \fi
  \listbegin
  \edef\activecount{\csname en@m\romannumeral\@ndentlev\endcsname}%
  \activecount=0 % initialize enum lev counter
  \seten@mlabwd \seten@mtoks % prepare the static part of the enum label
  \everypar{\enumeverypar}}
\def\endenumerate{\listend \advance\en@mlev-1\relax}

\def\itemmark{$\bullet$}
\def\dash{$-$}
\def\itemize{%
  \ifnum\@listnestlev=0 \@ndentlev=1\fi \listbegin
  \everypar{\hbox to\@ndentamt{\itemmark\hfil}\hangindent\@ndentamt}\ignorespaces}
\def\enditemize{\listend}

% outlining, from Turabian: 
%	LEVEL	COUNTER FMT
%	1	upper rom.		
%	2	upper alph.
%	3	arabic.
%	4	lower alph)
%	5	(arabic)
%	6	(lower alph)
%	7	lower rom)
\def\charequiv#1{%
  \count255=#1\advance\count255by`a\advance\count255-1 \char\count255}
\def\@utlinefmt{%
  \global\advance\activecount1
  \ifcase\@utlinelev
  \or\uppercase\expandafter{\romannumeral\@utlinei}.%
  \or\uppercase\expandafter{\charequiv\@utlineii}.%
  \or\number\@utlineiii.%
  \or\lowercase\expandafter{\charequiv\@utlineiv})%
  \or(\number\@utlinev)%
  \or(\lowercase\expandafter{\charequiv\@utlinevi})%
  \or\lowercase\expandafter{\romannumeral\@utlinevii})%
  \fi}
\def\outline{%
  \ifnum\@listnestlev=0 \@ndentlev=0\fi
  \listbegin \advance\@utlinelev1
  \edef\activecount{\csname @utline\romannumeral\@utlinelev\endcsname}%
  \activecount=0 % initialize enum lev counter
  \everypar={\hbox to\@ndentamt{\ignorespaces\@utlinefmt\hfil}%
             \hangindent\@ndentamt}} % define the label
\def\endoutline{\listend \advance\@ndentlev-1 }

\def\quickoutline{\noblackboxes\sloppy
  \gdef\<{\begin{enumerate}}\gdef\>{\end{enumerate}}}

\def\describeditemfont{\rm}
\def\describeditempunct{:}
\newbox\descbox \let\p@r\par \newif\ifdescriberunins
\def\description{\parindent0pt \advance\parskip0pt plus.1pt
  \advance\leftskip4\indention % \everypar{\hskip-4\indention}%
  \def\describe{%
    \setbox\descbox=\hbox\bgroup\describeditemfont
             \aftergroup\finish@describe
	     \def\par{\egroup}}}
\def\finish@describe{\leavevmode\hskip-4\indention
  \hboxtomin{4\indention}{\unhbox\descbox\unskip
			  \describeditempunct}%
  \ifhboxtominwaslong\newline\fi
  \let\par\p@r}
\def\enddescription{\endgraf\noindent}

% display environments
\def\@begdisplay{\ifhmode\endgraf\nointerlineskip\fi
  \displaytrue \displayspacing \narrower
  \parindent=0pt \parskip=2pt plus.1pt \@Above}
\def\@enddisplay{\endgraf\@Below\nointerlineskip\aftergroup\noindentnextpar}

\def\display{\@begdisplay}
\def\enddisplay{\@enddisplay}
\def\undisplay{\wider} % this should just cancel the effects of narrowing

\def\center{\hbox to\hsize\bgroup\hss\vbox\bgroup}
\def\endcenter{\egroup\hss\egroup}

\def\poetry{\@begdisplay\advance\rightskip0pt plus1fil
  \everypar{\hangindent2em}\obeylines\noindent\ignorespaces}
\def\endpoetry{\@enddisplay}

\newdimen\quoteindent
\def\quotebegin{\everypar{}}
\def\quote{\quotebegin
  \quoteindent\parindent 
  \@begdisplay\parindent\quoteindent
  \ifalwaysindent\indent\else\noindent\fi
  \ignorespaces}
\def\endquote{\@enddisplay}
\let\quotation\quote
\let\endquotation\endquote

% New environments for indexes and concordances

% indexwithdots (iwd)
\newif\ifiwdskip
\newbox\iwddotbox
\newdimen\indexeolspace \indexeolspace1em

\def\iwdsetbox{\setbox\iwddotbox=\hbox{\hss\kern1pt\nine.\kern1pt\hss}}
\def\iwdshoveright#1{%
  \unskip\penalty0\ifiwdskip\hskip1em\fi\hbox{}\nobreak
  \leaders\copy\iwddotbox\hskip0pt plus1filll
  \hbox{#1}}
\def\iwd@ix#1#2{%
  \setbox0=\hbox{#1\hbox to\indexeolspace{\hfil}#2}%
  \ifdim\wd0>\hsize\global\iwdskiptrue\else\global\iwdskipfalse\fi
  \noindent\iwdpar\strut#1\iwdshoveright{\strut#2}\endgraf\allowbreak}
\def\iwdpar{\hangindent\indention}
\def\indexhook{\singlespace}
\let\indexwithdotshook\indexhook

% index without leaders
\def\idx@ix#1#2{\noindent\hboxtomin{1.35\indention}{\strut#1\ }\relax
  \hangindent\indention\strut#2\endgraf\allowbreak}
\let\ixnodots\idx@ix

% concordance (actually same as idx@ix in current format)
\def\cnc#1#2{\leavevmode\hboxtomin{1.35\indention}{\strut\sum #1\ }\relax
  \hangindent1\indention\strut#2\endgraf\allowbreak}

\def\index{
%  \lineskip0pt \lineskiplimit\maxdimen \raggedright
  \let\ix\idx@ix\csname indexhook\endcsname}
\def\indexwithdots{
%  \lineskip0pt \lineskiplimit\maxdimen \raggedright
  \iwdsetbox\let\ix\iwd@ix\csname indexwithdotshook\endcsname}
\def\concordance{
  \lineskip0pt \lineskiplimit\maxdimen \raggedright
  \csname concordancehook\endcsname}
\def\endindex{}
\def\endindexwithdots{}
\def\endconcordance{}

%\input daletter % a special kind of environment

\endinput

%%%%%%%%%%%%%%%%%%%%%%%%%%%%%%%%%%%%%%%%%%%%%%%%%%%%%%%%%

$Log: env.tex,v $
Revision 0.7  2001/10/16 16:55:26  s
*** empty log message ***

Revision 0.6  2000/01/05 19:23:40  s
add \ifalwaysindent to enable overriding of suppression of indentation
after headings

parameterize font and punctuation of described items by new commands
\describeditemfont (default: \rm) and \describeditempunct (default: `:')

Revision 0.5  1996/05/27 16:18:45  s
*** empty log message ***

% Revision 0.4  1996/02/19  12:31:58  s
% generalize verbatim to use \verbatimfont rather than hardwired
% font
%
% Revision 0.3  1995/08/04  14:57:40  s
% make multiple paragraph \describe entries align correctly in description
% environment
%
% Revision 0.2  1995/06/11  19:16:28  s
% minor mods to begdisplay/enddisplay for new noindentnextpar stuff
%
% Revision 0.1  1994/12/12  03:56:05  s
% initial checkin
%
