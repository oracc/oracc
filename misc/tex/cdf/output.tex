% This file is part of the Digital Assyriologist.  Copyright
% (c) Steve Tinney, 1994, 1995.  It is distributed under the
% Gnu General Public License as specified in /da/doc/COPYING.
%
% $Id: output.tex,v 0.14 1996/09/02 09:40:18 s Exp sjt $

% Output routines for DA format
% =============================
\newif\iftwosided \newcount\fpoutputpending \newif\ifinnote \newif\ifdidMaybeEject

\countdef\pageno=0 \pageno=1 % first page is number 1
\newtoks\headline \newtoks\footline 
\newtoks\lefthdrinfo \newtoks\centrehdrinfo \newtoks\righthdrinfo
\newtoks\leftftrinfo \newtoks\centreftrinfo \newtoks\rightftrinfo
\let\centerhdrinfo\centrehdrinfo
\let\centerftrinfo\centreftrinfo
\newif\ifr@ggedbottom 
\newif\ifnumberthispage \numberthispagetrue
\newif\ifcountthispage  \countthispagetrue

\def\fullline{\hbox to\fullhsize} \let\lineline\line
\def\hfline#1{\hbox to0pt{#1\hss}}

\headline={\fullline{\rlap{\the\lefthdrinfo\hss}%
		     \fullline{\hss\the\centrehdrinfo\hss}%
		     \llap{\hss\the\righthdrinfo}}}
\footline={\fullline{\the\leftftrinfo\hss\the\centreftrinfo\hss\the\rightftrinfo}}
\newif\ifhdrsepline
\def\dohdrsepline{\kern3pt\hrule width\fullhsize height0.04em depth0pt \kern3pt}
\def\usehdrsepline{\hdrseplinetrue}
\def\nohdrsepline{\hdrseplinefalse}

\def\raggedbottom{\topskip 10\p@ plus60\p@ \r@ggedbottomtrue}
\def\normalbottom{\topskip 10\p@ \r@ggedbottomfalse} % undoes \raggedbottom
\def\folio{\ifnumberthispage\def\next{\f@li@}\else\let\next\relax\fi
  \next}
%  \ifnumberthispage\def\next{\f@li@}%
%  \else\def\next{\global\numberthispagetrue}\fi
%  \next}
\def\f@li@{\ifnum\pageno<\z@ \folioformat{\romannumeral-\pageno}%
    \else\folioformat{\number\pageno} \fi}
\let\f@lio\folio
\def\foliofont{\folioface\ten\rm}
\def\folioformat#1{\foliofont#1}
\def\nopagenumbers{\let\folio\relax}
\def\numberpages{\let\folio\f@lio}
\def\nonumberthispage{\global\numberthispagefalse}
\def\dontcountthispage{\global\countthispagefalse}
\def\advancepageno{\ifnum\pageno<\z@ \global\advance\pageno\m@ne
  \else\global\advance\pageno\@ne \fi} % increase |pageno|

%% THE WHOLE FOOTNOTE BUSINESS NEEDS REWRITING.

%% In particular, \foottopskip and \footlastlinedp should be 
%% automatically resized by reference to the size of the footnote
%% font, and the horrid code to detect split footnotes needs to be
%% redone (using \holdinginserts?).

% for footnotes called inside quotes, etc.
\newdimen\fnparindent 
\newdimen\savedindent % for pushing \parindent when doing above
\newinsert\footins 
% foottopskip must be set to same height as the foot strut
\newdimen\foottopskip
\newif\iffootsplit
\newbox\footbeginstrutbox 
\newbox\footendstrutbox 

\setbox\footbeginstrutbox=\hbox{\hfil}
\def\footbeginstrut{\copy\footbeginstrutbox}

\setbox\footendstrutbox=\hbox{\vrule height0pt depth2.85pt width0pt }
\def\footendstrut{\copy\footendstrutbox}

\foottopskip9pt
\def\footlastlinedp{2.85pt}
\def\footnotehook{\lineskip0pt \lineskiplimit0pt\relax}

\def\footsplitrule{\footnoterule}
\def\footnonsplitrule{\footnoterule}
\newdimen\footlastlinedp \footlastlinedp\maxdimen
\newcount\interfootpenalty \interfootpenalty0
\newskip\interfootskip \interfootskip0pt
\newdimen\fnoteparindent
\newskip\fnoteparskip

\def\footnote{\bgroup\f@otnote}
\def\f@otnote#1{\let\@sf\empty % parameter #2 (the text) is read later
  \ifhmode\edef\@sf{\spacefactor\the\spacefactor}\/\fi
  #1\@sf\vfootnote{#1}}
\def\vfootnote#1{\insert\footins\bgroup\everypar{}% in case a note is in an enumerate
  \interlinepenalty\interfootnotelinepenalty
  \topskip\foottopskip
  \splittopskip\foottopskip
  \splitmaxdepth\dp\strutbox \floatingpenalty\@MM
  \leftskip\z@skip \rightskip\z@skip \spaceskip\z@skip \xspaceskip\z@skip
  \fntextindent{#1}\fnotesize\notespacing
  \parindent\fnoteparindent \parskip\fnoteparskip
  \footbeginstrut\footnotehook\futurelet\next\fo@t}
\def\fo@t{\ifcat\bgroup\noexpand\next \let\next\f@@t
  \else\let\next\f@t\fi \next}
\def\f@@t{\bgroup\aftergroup\@foot\let\next}
\def\f@t#1{#1\@foot}
\def\@foot{\footendstrut\parfillskip0pt plus1fill\par
  \penalty\interfootpenalty\vskip\interfootskip\egroup\egroup}
\skip\footins=\bigskipamount % space added when footnote is present
\count\footins=1000 % footnote magnification factor (1 to 1)
\dimen\footins=8in % maximum footnotes per page

% footnote counting
\newcount\fnotecounter
\def\resetfnotecount{\global\fnotecounter=0 } \resetfnotecount
\def\setfnotecount#1{\global\fnotecounter=#1 }
\def\fnote{\unskip\bgroup\innotetrue\footnotehook
  \global\advance\fnotecounter by 1 % First bump the counter.
  % Now convert the current value of the counter into a superscripted numeral
  \f@otnote{$^{\rm\the\fnotecounter}$}}
\def\fnotesansmark{\unskip\bgroup\innotetrue\footnotehook
  \global\advance\fnotecounter by 1 % First bump the counter.
  % Now convert the current value of the counter into a superscripted numeral
  \f@otnote{}}
\def\fntextindent#1{%
  {\parindent0pt\leavevmode\hbox to\indention{\hfil#1\enspace}}}
%  \savedindent\parindent \parindent\fnparindent
%  \indent\llap{#1\enspace}\ignorespaces \parindent\savedindent }
\def\fnotenumber{\unskip\global\advance\fnotecounter1 $^{\the\fnotecounter}$}
\def\fnoterest{\unskip\bgroup
  \let\@sf\empty % parameter #2 (the text) is read later
  \ifhmode\edef\@sf{\spacefactor\the\spacefactor}\/\fi\@sf
  \vfootnote{$^{\the\fnotecounter}$}}

\newskip\enoteskip \enoteskip0pt plus1pt
\def\textenotemark#1{\let\@sf\empty
  \ifhmode\edef\@sf{\spacefactor\the\spacefactor}\/\fi
  \up{\rm#1}\@sf{}} % this space is a problem if people put note marks before punct
%  \,\up{\rm#1}\@sf}% \ } % this space is a problem if people put note marks before punct
\def\endenotemarkuncounted#1{\leavevmode\noindent\hbox to 2em{\up{#1}\hfil}\innotetrue}
\def\endenotemark#1{\global\advance\fnotecounter1
  \ifnum\fnotecounter=#1 \else\errmessage{End note number doesn't match internal count}\fi
  \leavevmode\noindent\hbox to 2em{\up{#1}\hfil}\innotetrue}
\let\enotemark\textenotemark \let\fnmark\textenotemark
\def\enoteend{\endgraf\innotefalse\vskip\enoteskip}
\def\endnotes{\endnotehdr\let\enotemark\endenotemark\global\fnotecounter0 }
\def\endendnotes{}
\let\endnoteshere\relax
\def\endnotehdr{\newpage\heading Endnotes\par\indentnextpar}

% When writing files for the DA format only \note should be used, as 
% Pretex controls whether footnotes or end notes are produced according
% to the -e switch
\let\note\fnote

\newinsert\topins
\newif\ifp@ge \newif\if@mid
\newif\if@reallymid

\def\topinsert{\@midfalse\p@gefalse\@ins}
\def\midinsert{\@midtrue\@ins}
\def\MIDinsert{\@midtrue\@reallymidtrue\@ins}
\def\pageinsert{\@midfalse\p@getrue\@ins}

\skip\topins=\z@skip % no space added when a topinsert is present
\count\topins=1000 % magnification factor (1 to 1)
\dimen\topins=\maxdimen % no limit per page
\def\@ins{\par\begingroup\setbox\z@\vbox\bgroup} % start a \vbox
\def\endinsert{\egroup % finish the \vbox
  \if@mid\if@reallymid\else
    \dimen@\ht\z@ \advance\dimen@\dp\z@ \advance\dimen@12\p@
    \advance\dimen@\pagetotal \advance\dimen@-\pageshrink
    \ifdim\dimen@>\pagegoal\@midfalse\p@gefalse\fi\fi\fi
  \if@mid %\bigskip
	  \box\z@
	  \penalty-200
	  %\bigbreak \message{mid true}
  \else%\message{mid false}%
    \insert\topins{\penalty100 % floating insertion
    \splittopskip\z@skip
    \splitmaxdepth\maxdimen \floatingpenalty\z@
    \ifp@ge \dimen@\dp\z@
    \vbox to\vsize{\unvbox\z@\kern-\dimen@}% depth is zero
    \else \box\z@\nobreak
%\bigskip
\fi}\fi\endgroup}

\def\pagebody{{\vbadness10000\vbox to\vsize{
  \ifvoid\xrefinsert\else
     \rlap{\kern30pc \vbox to0pt{\box\xrefinsert\vss}}\fi
  \boxmaxdepth\maxdepth \pagecontents}}%
  \ifsources\global\moresourcestrue\fi}

\def\makeheadline{\vbox to\z@{\vskip-\headersep
  \hfline{\vbox to8.5\p@{}% this dimen is an artefact of Plain TeX and troubles
			% me, but I can't see how to parameterize it yet. It
			% doesn't seem terribly important in any case.
  \the\headline}\ifhdrsepline\dohdrsepline\fi
  \vss}\nointerlineskip}
\def\makefootline{\baselineskip\footersep \the\footline}
\def\dosupereject{\ifnum\insertpenalties>\z@ % something is being held over
  \line{}\kern-\topskip\nobreak\vfill\supereject\fi}

\newskip\fnotebelowruleskip \fnotebelowruleskip3pt plus2pt minus1pt

%\def\footnoterule{\kern-3\p@ \hrule width 2truein \kern 2.6\p@ % the \hrule is .4pt high
%  \vbox{}\vskip\fnotebelowruleskip\relax}

% New footnoterule def from HRW's hebtex.tex
\def\footnoterule{\kern-3\p@
  \ifhfnoterule \line{\hfil\hbox to 2truein{\leaders\hrule\hfil}}%
  \else \hrule width 2truein
  \fi
  \kern 2.6\p@ % the \hrule is .4pt high
  \vbox{}\vskip\fnotebelowruleskip\relax}

%%%%%%%%%%%%%%%%%%%%%%%%%%%%%%%%%%%%%%%%%%%%%%%%%%%%%%%%%
% multicolumn output routines
%%%%%%%%%%%%%%%%%%%%%%%%%%%%%%%%%%%%%%%%%%%%%%%%%%%%%%%%%

\newif\ifmc@ctive
\newbox\partialpage		\newbox\whatsleft
\newdimen\intercolgutter	\intercolgutter=.1in
\newdimen\savedvsize
\newcount\tmpct			\newcount\tmpctb
\newcount\@numcols
\newif\ifgutterrules
\def\gutterrule{\ifgutterrules\vrule width0.04em\relax\hfil\fi}
\newif\ifcolnobalanceatend

% External interface; user says @begin{2column}...@end{2column} etc.
\expandafter\def\csname 2column\endcsname{\@beginmcol2}
\expandafter\def\csname 3column\endcsname{\@beginmcol3}
\expandafter\def\csname 4column\endcsname{\@beginmcol4}
\expandafter\def\csname 5column\endcsname{\@beginmcol5}
\expandafter\def\csname 6column\endcsname{\@beginmcol6}
\expandafter\def\csname end2column\endcsname{\@endmcol}
\expandafter\def\csname end3column\endcsname{\@endmcol}
\expandafter\def\csname end4column\endcsname{\@endmcol}
\expandafter\def\csname end5column\endcsname{\@endmcol}
\expandafter\def\csname end6column\endcsname{\@endmcol}

\expandafter\def\csname 2columnbox\endcsname{\@beginmcolbox2}
\expandafter\def\csname 3columnbox\endcsname{\@beginmcolbox3}
\expandafter\def\csname 4columnbox\endcsname{\@beginmcolbox4}
\expandafter\def\csname 5columnbox\endcsname{\@beginmcolbox5}
\expandafter\def\csname 6columnbox\endcsname{\@beginmcolbox6}
\expandafter\def\csname end2columnbox\endcsname{\@endmcolbox}
\expandafter\def\csname end3columnbox\endcsname{\@endmcolbox}
\expandafter\def\csname end4columnbox\endcsname{\@endmcolbox}
\expandafter\def\csname end5columnbox\endcsname{\@endmcolbox}
\expandafter\def\csname end6columnbox\endcsname{\@endmcolbox}

\newif\iffp@mc

%%%%%%%%%%%%%%%%%%%%%%%%%%%%%%%%%%%%%%%%%%%%%%%%%%%%%%%%%
% Facing page routines
%%%%%%%%%%%%%%%%%%%%%%%%%%%%%%%%%%%%%%%%%%%%%%%%%%%%%%%%%
%
% These routines are simple in the sense that they read all of the text
% to be faced up before sending any of it to the page builder. This means
% that for longer examples you'll need a 32 bit/big TeX to process your
% files. 
%
% The principle is simple: all of the left page material is read in,
% after which all of the right page material is read in. Then pages are
% lopped off of each box in turn, padded with whitespace so they're the
% same depth and sent to the page builder (the page builder
% may be in multi-column mode at this point, so transliteration and 
% translation can actually be prepared in facing columns as well as
% facing pages). The internal format of the left and right page boxes
% is a simple series of vboxes each of which must represent a unit for 
% alignment purposes. [Similar, parallel, lists of vboxes may be present in
% the variant and footnote boxes, and will be taken into account 
% appropriately. (not yet implemented)]
%
% The only slightly complicated part is choosing where to lop off the pieces,
% as it is desirable to account for footnotes and variant insertions. These
% may be simply placed at the bottom of the page, or balanced across the 
% facing pages.
%
% The present first attempt at facing page macros does not take account of
% such inserts.
%
% General Structure:
% ==================
%
%	\fpbegin	: initialize left page collection
%	\fpswitch	: terminate left and initialize right page collection
%	\fpend		: terminate right page collection and send amassed 
%			  material to page builder
%
% These commands also have external forms:
%	\facingpage
%	\switch
%	\endfacingpage
% to conform to the structure of begin...end environments.
%
\newif\iffp@ctive	\newif\iffpleftover@	\newif\iffpspaceatend
\newif\iffpnopresplit
\newbox\fpleftb@x	\newbox\fpleftp@ge	\newbox\fpleftb@@
\newbox\fprightb@x	\newbox\fprightp@ge	\newbox\fprightb@@
\newbox\fppreleftb@x	\newbox\fpprerightb@x
\newdimen\fppagegoal	\newdimen\fppagetotal	\newdimen\vf@zz
\newdimen\fpb@xsize	\newskip\fpskip	\fpskip5pt plus1fil minus1pt
\newdimen\fpshrink	\fpshrink0pt
\newdimen\fppreskip	\fppreskip5pt
\newdimen\fp@lreadysize % size of stuff already on page when fp begins
\newcount\fpnumsof@r
\newdimen\fpb@xht	\newdimen\fpb@xdp

\def\fpbegin{
  \ifmc@ctive
    \def\next{\fp@lreadysize\ht\partialpage \advance\fp@lreadysize\dp\partialpage}
  \else
    \def\next{\fppre@}
  \fi
  \next
  \fp@ctivetrue
  \setbox\fpleftb@x\vbox\bgroup\box\fppreleftb@x}

% When this macro is invoked there may already be material on the
% page. The solution is simply to stash that material in two registers,
% \fppreleftb@x and \fpprerightb@x, which are always inserted at the
% start of a new left/right box. The material is split if possible, and
% if the user hasn't set \fpnopresplittrue. Space equivalent to 
% \fppreskip is inserted between these initial boxes and the main left/
% right pages
\def\fppre@{
  \ifdim\pagetotal > 0pt
    \toks0\output \output{\fppre@output} \eject \output\toks0
  \else
    \iftwosided\ifodd\pageno\else\newpage\fi\fi
  \fi
  \fp@lreadysize=\pagetotal}
\def\fppre@output{
  \global\setbox\fppreleftb@x=\vbox{}
  \global\setbox\fpprerightb@x=\vbox{}
  \setbox0=\vbox{\unvbox255} % retrieve prev stuff from main list
  \ifdim\ht0>0pt \def\next{\fpsetpreb@xes}\else\let\next\relax\fi
  \next}
\def\fpsetpreb@xes{{\vbadness10000 \vfuzz\maxdimen
  \iffpnopresplit
    \global\setbox\fppreleftb@x=\vbox{\unvbox0}
  \else
    \global\setbox\fppreleftb@x=\vsplit0 to.5\ht0 
    \global\setbox\fppreleftb@x=\vbox{\unvbox\fppreleftb@x}
    \global\setbox\fpprerightb@x=\vbox{\unvbox0}
  \fi
  \ifdim\ht\fpprerightb@x>\ht\fppreleftb@x\def\next{\fpbalanceb@xes}
  \else\let\next\relax\fi
  \next}
  \fpfinishb@xes}
\def\fpbalanceb@xes{
  \loop
    \setbox0=\vsplit\fpprerightb@x to0pt
    \global\setbox\fppreleftb@x=\vbox{\unvbox\fppreleftb@x\unvbox0}
  \ifdim\ht\fprightb@x > \ht\fppreleftb@x \repeat
}
\def\fpfinishb@xes{
  \global\setbox\fppreleftb@x=\vbox{\unvbox\fppreleftb@x\vbox to\fppreskip{\vfil}}
  \global\setbox\fpprerightb@x=\vbox{\unvbox\fpprerightb@x\vbox to\fppreskip{\vfil}}
}
\def\fpswitch{\egroup\setbox\fprightb@x\vbox\bgroup\box\fpprerightb@x}

\def\fpend{\egroup\fpformat\fp@ctivefalse
  \ifmc@ctive\else\box\fpleftp@ge\global\fpoutputpending=1\relax\fi}

\def\fpoutput{
	\vbox to\ht255{\unvbox255\vfill}\penalty0
	\box\fprightp@ge\penalty\outputpenalty
	\global\fpleftover@false
	\global\output{\plainoutput}}

\newcount\fpflushn@m

% define syntactic sugar
\let\facingpage\fpbegin \let\switch\fpswitch \let\endfacingpage\fpend

%%%%%%%%%%%%%%%%%%% end of facing page macros %%%%%%%%%%%%%%%%%%%%%%%%%%

\def\resetoutputvariables{%
  \global\countthispagetrue
  \global\chapterpagefalse
  \global\numberthispagetrue}

\newif\ifmakevsizefull

\def\plainoutput{%
  \ifnum\mc@added>0 \global\mc@nopageyetfalse\fi
  \global\didMaybeEjectfalse
  \shipout\vbox{\makeheadline\pagebody\makefootline}%
  \ifcountthispage\def\output@next{\advancepageno}\else\let\output@next\relax\fi
  \output@next
  \resetoutputvariables
  \ifmakevsizefull\global\makevsizefullfalse\global\vsize\textheight\fi
  \ifnum\outputpenalty>-\@MM \else\dosupereject\fi
  \ifnum\fpoutputpending=1 \global\advance\fpoutputpending-1
    \box\fprightp@ge
  \fi
  \flushpageinserts}

\output{\plainoutput}
\outer\def\bye{
  \ifnum\fpoutputpending=1
    \line{}\kern-\topskip\nobreak\vfill\eject
  \else
    \par
  \fi
  \vfill\supereject\End}

\newdimen\@coltarget

\input boxincol
\input mcolnote
\input multicol

\endinput

%%%%%%%%%%%%%%%%%%%%%%%%%%%%%%%%%%%%%%%%%%%%%%%%%%%%%%%%%

$Log: output.tex,v $
Revision 0.14  1996/09/02 09:40:18  s
Doh! Remove debug \show statement from \plainoutput

Revision 0.13  1996/09/02 09:39:20  s
modify \fpend, \plainoutput and \bye to ensure that final bits of
  facing-page stuff are emitted correctly

Revision 0.12  1996/08/22 01:31:18  s
add \let\fnmark\textenotemark for backwards compatibility with
earlier DA macro packages

Revision 0.11  1996/08/10 11:31:30  s
merge HRW's redefinition of footnoterule for LR-adjustability

Revision 0.10  1996/05/27 16:19:15  s
*** empty log message ***

% Revision 0.9  1996/04/05  12:06:35  s
% first hack at stripping out superseded stuff that's now in
% multicol and boxincol---needs more work, and possibly more
% modularization (e.g., a facepage.tex module)
%
% Revision 0.8  1996/02/17  23:41:34  s
% revisions to facing page macros
%
% revisions to midinsert etc NOTE: THEY ADD NO SPACE NOW---PROBABLY
% NEEDS PARAMETERIZING
%
% enhanced footnote code to support differentiating between split and
% non-split notes NOTE: THIS SUPPORT IS A NASTY KLUDGE
%
% Revision 0.7  1995/10/27  03:16:56  s
% add \footnotehook to \fnote and \footnote
%
% Revision 0.6  1995/09/21  01:00:03  s
% fix \moresources magic to make source header blocks carry
% over from page to page
%
% Revision 0.5  1995/06/15  10:58:17  s
% fixed nasty sleeper bug with \blankpage: if \nopagenumbers was in effect, \folio
%       never got called, which meant that \nonumberthispage didn't get switched
%       off until pages later.  Now there is a command \resetoutputvariables that
%       every output routine should call after the actual page make up has been
%       performed.
%
% Revision 0.4  1995/06/15  03:50:27  s
% \ifchapterpage needs to be set false after an output routine is called;
%    added it in \mcolcolumnout
%
% Revision 0.3  1995/06/14  15:00:23  s
% fixed multi-column balancing (finally!?) by removing spurious \vbox{} from
% \@endmcol and implementing the \loop\unskip\if\lastskip>0pt\repeat trick
% in \balancecolumns
%
% Revision 0.2  1995/06/14  13:24:31  s
% \ifchapterpage needs to be set false after an output routine is called
%
% Revision 0.1  1994/12/12  03:56:05  s
% initial checkin
%
